% !TEX program = pdflatex
% !BIB program = biber

%% One can optionally have all this inside a separate setup.tex
% !TEX program = pdflatex
% !BIB program = biber

% !TEX root = main.tex
\documentclass[10pt, a4paper]{article}
% \documentclass[12pt, a4paper, oneside]{memoir}
% \chapterstyle{veelo}

%% Sets page size and margins
\usepackage[a4paper,top=3cm,bottom=2cm,left=3cm,right=3cm,marginparwidth=1.75cm]{geometry}
%\usepackage[left=2cm,top=2cm,bottom=2cm,bindingoffset=1cm]{geometry}

\usepackage[utf8x]{inputenc}
\usepackage{graphicx}
\usepackage{float}
\usepackage{imakeidx}
\usepackage{amsmath}
\usepackage[colorlinks=true, allcolors=blue]{hyperref}
\usepackage{graphicx}
\usepackage{float}
\usepackage{imakeidx}
\usepackage{amsmath}
\usepackage{url}
\usepackage[export]{adjustbox}
\usepackage{subcaption}

%% Useful packages
\usepackage[colorinlistoftodos]{todonotes}
\usepackage[T1]{fontenc}

%% Language and font encodings
\usepackage[english]{babel}

%% Define a few colours to be used throughout the document
\usepackage{tikz,xcolor}
\definecolor{TextColor}{HTML}{000000}
\definecolor{SideColorDark}{HTML}{000000}
\definecolor{MainColor}{HTML}{0000FF}
\definecolor{OppositeColor}{HTML}{FF0000}
\definecolor{HighlightColor}{HTML}{FFFF00}


%% Code block style
%  Load the \ttfamily font
\usepackage[T1]{fontenc}
\usepackage[scaled]{beramono}

%  Format code blocks
\usepackage{listings}
%  Change caption name
\renewcommand*{\lstlistingname}{Code block}
\captionsetup[lstlisting]{margin=0cm,format=hang,font=small,format=plain,labelfont={bf,up},textfont={it}}
%  Style
\lstset{
  showstringspaces=false,
  formfeed=\newpage,
  commentstyle=\itshape,
  backgroundcolor=\color{gray!5},
  breakatwhitespace=false,         % sets if automatic breaks should only happen at whitespace
  breaklines=true,                 % sets automatic line breaking
  captionpos=b,                    % sets the caption-position to bottom
  commentstyle=\color{gray},    % comment style
  escapeinside={\%*}{*)},          % if you want to add LaTeX within your code
  keepspaces=true,
  numbersep=2mm,                   % how far the line-numbers are from the code
  showspaces=false,
  showstringspaces=false,
  showtabs=false,
  stepnumber=1, numberfirstline=false,
  basicstyle=\linespread{1}\footnotesize\ttfamily,
  keywordstyle=\bfseries\color{MainColor},
  stringstyle=\itshape\color{OppositeColor},
  numberstyle=\footnotesize\ttfamily\color{gray},
  numbers=left,xleftmargin=4mm,framexleftmargin=0mm,xrightmargin=0mm,
  frame=top,frame=bottom,
}

\title{Insert title here}
\author{Daniel Robinson\\18361137}

\pagenumbering{roman}
\setlength\parindent{0pt}

\begin{document}
% \maketitle
    \begin{titlepage}
        \begin{center}
            \vspace*{1cm}
            
            \begin{figure}
			\centering
            \includegraphics[scale=2]{images/UScrest-top.jpg}
            \end{figure}
            
            \huge
            \textbf{Normalized Differential Vegetation Index Mapping}

            \large            
            \vspace{2.5cm}

            \textbf{Daniel Robinson\\18361137}

            \vspace{2.5cm}    
            
            \textbf{Report submitted in partial fulfilment of the requirements of the module Project (E) 448
            for the degree Baccalaureus in Engineering in the Department of Electrical and Electronic
            Engineering at the University of Stellenbosch}
            
            \vspace{4cm} 
            
            \textbf{STUDY LEADER: Corné van Daalen\\DATE: November 2017}
            
            
        \end{center}
\end{titlepage}

\setcounter{secnumdepth}{-2}% default for "report" is 2

\newpage
\chapter{Declaration of own work}

I, the undersigned, hereby declare that the work contained in this report is my own original work
unless indicated otherwise.

\vspace{2.5cm}   

Signature: \hrulefill

\hspace*{0mm}\phantom{Signature: }Daniel Robinson

\hspace*{0mm}\phantom{Signature: }BEng Electrical \& Electronic

\vspace{1cm}   

Date:\space\space\space\space\space\hrulefill

\chapter{Abstract}

This report presents the design and analysis necessary for an autonomous system to acquire beneficial agricultural data within a desired environment.\\

Various implementations are compared, before a novel design is presented.\\

An aerial drone is designed and built using predominantly local supplies. Images are acquired, calibrated, processed and mapped before an NDVI is performed. The system is quite capable and cost-effective compared to existing solutions. However, further investigation is required in mapping and NDVI calibration for consistency.

\chapter{Uitreksel}

Hierdie verslag toon die nodige ontwerp en ontleding aan om nuttige landbou data van 'n uitverkose omgewing deur middel van 'n selfbesturende sisteem te verkry.\\

Verskye toepassings is met mekaar vergelyk voordat 'n unieke ontwerp aangebied is.\\

'n Onbemande hommeltuig is ontwerp en gebou deur hoofsaaklik van plaaslike onderdele gebruik te maak. In die proses is foto's verkry, gekalibreer, verwerk en gekarteer voordat 'n genormaliseerde differensiële plantegroei indeks (GDPI) bereken is. Die stelsel is geskik en koste effektief in vergelyking met huidige komersiële oplossings.\\

Vir konsekwensie word verdere ondersoek egter in kartering en kalibrasie van die GDPI verlang.

\chapter{Acknowledgements}

I would like to thank the following people for their support:
\begin{itemize}
    \item Louw Hopley, for giving birth to the idea.
    \item Alan Knott-Craig, for introducing me to Chris Antoniesson, for we went on a flight test together which returned very useful data.
    \item My family, who have been amused and for their unwavering love and support.
    \item Dr Corné van Daalen, for his guidance and support throughout the course of this project.
\end{itemize}

%\frontmatter
\cleardoublepage
\tableofcontents

\clearpage
\label{listoffigures}
\addcontentsline{toc}{chapter}{\listfigurename}
\listoffigures

%\clearpage
%\addcontentsline{toc}{chapter}{\listtablename}
%\label{listoftables}
%\listoftables

\chapter{List of Symbols}
\begin{table}[H]
\centering
\begin{tabular}{l|r|r}
Symbol & Description & Units\\\hline\\
$F$ & Fundamenal Matrix & 3x3\\
$R$ & Rotation Matrix & 3x3\\
$T$ & Translation Matrix & 3x3\\
\end{tabular}
%\caption{\label{tab:symb}Table depicting list of symbols. (Will probably not have this caption)}
\end{table}

\chapter{List of Abbreviations}
\begin{table}[H]
\centering
\begin{tabular}{l|r}
Abbreviation & Description \\\hline
UAV & Unmanned Aerial Vehicle \\
NDVI & Normalised Differential Vegetation Index\\
IR & Infra Red\\
PAR & Photosynthetically Active Radiation\\
PCB & Printed Circuit Board\\
CSI & Camera Serial Interface\\
SBUS & Serial Bus ?\\
GPIO & General Purpose In Out\\
LUT & Look Up Table\\
FHSS & Frequency Hopping Spread Spectrum\\
NLOS & Non Line of Sight\\
SDR & Software Defined Radio\\
\end{tabular}
%\caption{\label{tab:abbr}Table depicting list of abbreviations. (Will probably not have this caption)}
\end{table}

%\chapter{Matrix Notation}
%
%(Still to be expanded on).

\setcounter{secnumdepth}{2}

\chapter{Introduction}

\pagenumbering{arabic}

There is currently a water shortage crisis in South Africa, as well as in other parts of the world, potentially due to global warming. One of the sectors that it affects is agriculture. The water shortage puts added stress on farmers who may already be watering their crops non-uniformly; including the application of fertilizer and pesticides. Even plant disease contributes to the problem.\\

To mitigate such problems, farmers would typically use their expertise in observation from the ground. It can be a time consuming task, especially over areas of a few square kilometres. If static sensor data is used, the readings may be too sparse due to the exponential cost and complexity of multiple static sensors. These sensors require maintenance, post-processing and an estimated interpretation of its data. This does not exclude hand-held readings, which may only be able to point out some areas that need more attention. Perhaps a simpler, large-scale method of observing such changes at once can be used.\\

The human mind is an exceedingly powerful tool to analyse great complexities around the world. Computers may not be as creative, but they boast vast processing power and numerical solving (cite source). Offloading computations and automation better suited for computers from human thought leaves the mind free to do tasks better suited for itself.\\

To observe an area in the digital world, one can observe signals from the electromagnetic spectrum. Although there exists suitable equipment today, it remains yet inaccessible due to the complexity and relatively high cost of equipment. (cite an example) This project looks at a way to bridge this gap somewhat.

\section{Purpose of the project}

The project is conducted to investigate the core elements required to map vegetation via infrared analysis techniques, which can thus potentially determine the impact of human and environmental factors such as diseases, erosion, variances in fertiliser and water application.\\

In order to be deemed successful the project needs to investigate observable parameter(s) beneficial to agriculture within large areas.

\section{Problem Statement}\

To analyse large-scale agricultural areas and produce meaningful data, imagery is typically collected and mathematical algorithms are performed on the data. There are a number of algorithms, with a popular one being the Normalised Differential Vegetation Indice, or NDVI for short. This particular one will be the focus of the project. If successful, it opens up the possibility of investigating other useful indices, perhaps even for the mining industry if this project continues.\\

Cameras will be used to acquire data. Filters will be used to isolate bandwidths required by the vegetation indices.\\

To monitor large areas in a short period of time, one needs height, which can be achieved by the modern invention of flight. In more practical terms, we can use vehicles such as hot air balloons, airstats, planes, helicopters and the like. We can classify these vehicles into manned and unmanned aerial vehicles. Unmanned vehicles have significantly less safety precautions, regulations, cost, and are less subject to wear and tear, meaning more flights (cite).\\

In brief, the idea is to develop a nadir\footnote{Downwards facing} observation platform to map areas from height, to determine observation equipment, to process sampled data and draw conclusions, and to develop control tests to verify / prove conclusions\\

The implementation of autonomous systems can be costly. It also depends on what type of UAV is used. Unmanned aeroplanes  are fast. With enough height, the effects of motion blur tend to zero. On the other hand, they are significantly complex, and although it would be ambitious, it would be impractical for a project such as this. Aerostats\footnote{Lighter than air aircraft that gains its lift through the use of a buoyant gas} and hot air balloons\footnote{Unpowered aerostat which has no means of propulsion and is usually tethered} can stay up for long periods of time, but are impractical due to the limited uses, and difficulty in moving such a vehicle once in the air. Drones on the other hand are inexpensive, and can take accurate and versatile high resolution images, with less risk to the camera equipment than a plane.\\

Thus, a drone is designed and built to demonstrate proof of concept of a more cost effective model compared to solutions that exist in the market today.\\

Various assumptions are made in order to realise the solution of the problem. Besides assumptions regarding the environment in which the system will operate, it will also provide a starting basis for design synthesis. NDVIs are calculated predominantly in the farming sector, the values vary according to the age of the plant, sunlight or clouds will affect the values, and the cameras will be simple to calibrate.

\section{System Overview}

Figure \ref{fig:scenario} illustrates a typical scenario where the system is used.\\

\begin{figure}[H]
\centering
\includegraphics[scale=0.4]{images/drone_ndvi_scenario.png}
\caption{Block diagram system overview}
\label{fig:scenario}
\end{figure}

An implementation process of the system is illustrated in Figure \ref{fig:overview}. These processes will be expanded on and explained in the relevant chapters.

\begin{figure}[H]
\centering
\includegraphics[scale=0.6]{images/thesis_overview.png}
\caption{Block diagram implementation process}
\label{fig:overview}
\end{figure}

Various drones, cameras, filters and processing techniques will be presented, as well as optimal choices depending on available resources.\\

A literature study will be investigated in Chapter 2, with drone design and construction in Chapter 3. Image acquisition and calibration will be handled in Chapter 4, and image processing in Chapter 5. Lastly, system integration and testing will be discussed in Chapter 6.
\label{chapter:intro}
\chapter{Literature Study}

The literature study includes current knowledge about the subect, is evaluated, and consists of fundamental concepts to note.\\

Infrared light extends from the nominal red edge of the visible spectrum at 700 nanometers (430 THz) to 1 mm (300 GHz) \cite{ir_wiki}. The infrared light to be used for NDVI analysis is between 700 and 1000 nm due to silicon response \cite{ir_wiki}, and is known as near infrared light (NIR).

\begin{figure}[H]
\centering
\includegraphics[scale=0.35]{images/ir_spectrum.png}
\caption{Wavelengths of light \cite{ir_spectrum}}
\label{fig:ir_spectrum}
\end{figure}

The normalized difference vegetation index (NDVI), which forms the base of this project, is a simple graphical indicator that can be used to analyze remote sensing measurements, typically but not necessarily from a space platform\footnote{Aerial footage will always be higher resolution, with each pixel capturing $cm^2$ as opposed to $m^2$}, and assess whether the target being observed contains live green vegetation or not\cite{ndvi_wiki} as in Figure \ref{fig:ndvi_british}.

\begin{figure}[H]
\begin{subfigure}{0.5\textwidth}
\centering
\includegraphics[scale=0.25]{images/NDVI_062003.png}
\caption{June 2003}
\end{subfigure}
\begin{subfigure}{0.5\textwidth}
\centering
\includegraphics[scale=0.25]{images/NDVI_102003.png}
\caption{October 2003}
\end{subfigure}
\caption{Average NDVI over the British Isles \cite{ndvi_wiki}}
\label{fig:ndvi_british}
\end{figure}

The NDVI is calculated using the following formula for each pixel:\\

\begin{equation}\label{eq:ndvi}
{\displaystyle{\mbox{NDVI}}={\frac {({\mbox{NIR}}-{\mbox{Red}})}{({\mbox{NIR}}+{\mbox{Red}})}}}, \in [-1.0, 1.0]
\end{equation}

where red and NIR stand for the spectral reflectance measurements acquired in the red (visible) and near-infrared regions, respectively.

\begin{figure}[H]
\centering
\includegraphics[scale=0.5]{images/chlorophyll.jpg}
\centering
\caption{PAR and absorption spectrum \cite{ndvi_wiki}}
\label{fig:chlorophyll}
\end{figure}

In Figure \ref{fig:chlorophyll}, typical photosynthetic action in the photosynthetically active radiation (PAR) spectral region for live green plants is shown, beside absorption spectra for chlorophyll and carotenoids. Solar radiation is absorbed within this region as a source of energy. \\

Leaf cells re-emit solar radiation in the near-infrared region because the photon energy at wavelengths longer than 700 nm is not large enough to synthesize organic molecules, even though it comprises approximately half of the total incoming solar energy. If it were absorbed, it would only result in damage and overheating of the plant.\\

The NDVI is similar to a mere ratio of infrared light to visible light, except the ratio is not normalised, therefore infinite values can exist.\\

The rationale behind the NDVI is that one can exploit the strong differences in plant reflectance to determine their spatial distribution.\\

The launch of the Sputnik 1 by the Soviet Union in 1957 sparked an interest in meteorological satellites to improve weather forecasting. NDVIs, introduced in the early seventies by the ERTS\footnote{Earth Resources Technology Satellite, or Landsat 1}, remain popular today. \\

Today, for example, one can expect to pay about \$1600 per fortnightly 0.5m resolution photo from the Pléiades satellite constellation\footnote{Very-high-resolution optical earth-imaging satellites via Satellite Imaging Corporation}. There are other sensors with similar pricing, including the WorldView, GeoEye, Kompsat, Quickbird, Gaofen etc.\\Aerobotics, based in Cape Town have quite a cost effective solution, with weekly footage at 500 ZAR per month, however at a low resolution of 10m per pixel. They also charge 80 ZAR per hectare for drone surveys, and 20 ZAR per hectare\footnote{Worldwide average 59.4 Ha\cite{farm_size}, South Africa between 427 - 5799 Ha \cite{farm_size_sa}} if one uploads one's own data. Solutions like these show that it could be beneficial for a farmer to have their very own automated solution.\\

Since the proliferation of UAVs, there has been an increased effort to find more cost effective solutions to commercial applications, not excluding NDVI analysis. Hypothetically, if an inexperienced farmer were to perform NDVIs of their land by themselves, there does exist online platforms to process the imagery. However, imagery may not necessarily be calibrated, and distortions in the lens may create an inaccurate result. Also, the upfront costs of the hardware required may be too much. Typical drones like the Phantom series cost around \$1000-2000. \\Cameras and pricing include:
\begin{itemize}
	\item Parrot Sequoia, 5 cameras, \$3500 \cite{sequoia}
	\item Mapir Survey2, single camera, \$500 \cite{mapir}
	\item AgroCam Pro, dual cameras, \$575 \cite{agrocam}
\end{itemize}

There are other vegetation indices such as the Enhanced vegetation index in equation \ref{eq:evi},

\begin{equation}\label{eq:evi}
EVI=G\times {\frac  {(NIR-RED)}{(NIR+C1\times RED-C2\times Blue+L)}}
\end{equation}

where NIR, blue and red refer to atmospherically corrected Raleigh and ozone absorption surface reflectances, L refers to background adjustment in canopies that address non-linear, differential NIR, and the coefficients C1, C2 uses the blue band to correct for aerosol influences in the red band. The EVI, as opposed to the more chlorophyll sensitive NDVI, is more responsive to structural canopy variations, including LAI (leaf index area), physiognomy\footnote{External appearance, and growth forms of dominant taxa} and architecture \cite{evi}. Since it has more of a satellite application, this exists outside the scope of the project.\\

Another example is the Normalized difference water index as in equation \ref{eq:ndwi},

\begin{equation}\label{eq:ndwi}
{\displaystyle {\mbox{NDWI}}={\frac {(Xnir-Xswir)}{(Xnir+Xswir)}}}\quad or\quad {\displaystyle {\mbox{NDWI}}={\frac {(Xgreen-Xnir)}{(Xgreen+Xnir)}}}
\end{equation}

where short-wave infrared (SWIR\footnote{Water absorption increases significantly at 1450 nm, within SWIR wavelength 1.4-3 µm}) wavelengths are used to monitor changes in water content of leaves, and green and NIR wavelengths monitor changes realted to water content in water bodies. The former is useful for vegetation, and the latter is useful to detect flooding or changes in water level \cite{ndwi}. SWIR cameras are expensive, however (cite source).\\

Although there exist other vegetation indices, it should be inductive that they can be proven, if in this project the NDVI is successful.\\

For NDVIs, it should be noted that there are many who motivate that a cheap conversion of a normal digital camera (by replacing the NIR filter with a blue or red filter) will show satisfactory results. Professionals (cite source) will generally advise one to avoid such methods since one can never remove the cross-channel interference (see section \ref{sec:intrinsic}, unless one has dual cameras.\\

The project focuses on a dual/stereo camera setup, to minimize cross-channel interference. However, several problems arise such as distortion, differences in focal length, rotation and translation between the cameras, and so forth.\\

There exist techniques to solve these problems, with an initial step to calibrate the cameras using corresponding points in a 3D plane (cite). Then, images are stereo-rectified (cite) and undistorted (cite). The common area may still differ in focal length, motion (add more detail here) and for this reason, stereo matching (cite) is used to align the images as best as possible. Only then can an NDVI be calculated for each matching pixel.


\label{chapter:literature}
\chapter{Drone Design and Construction}

A drone is designed and constructed. The goal is to be cost-effective and to source local supplies. It should be noted that pre-built drones like the DJI Phantom and Inspire cost significantly more\footnote{Easily 100\% to 200\% more}. This build, including the cameras costs about 10000 ZAR.

\section{Drone}

\subsection{Mechanical subsystem design}

The drone will be medium sized, with a payload of 500-1000g. It would be ideal to have a good thrust to weight ratio of 3:1.

\subsubsection{Frame}

There are many materials and configurations for a drone. The configurations include 3, 4, 6, 8 arms or more, with single or contra-rotating propellers on each arm\footnote{Two propellers aligned vertically and spinning opposite to each other for upwards thrust}. \\

In aerial cinematography, it would require a stiff but less brittle frame as in Figure \ref{fig:hex} to provide a smooth and stable flight. They also need to be large enough to hold the cameras needed for this professional activity. Lastly, the frames should be supportive of tall landing gear \cite{frame}. Due to the larger overall mass, the frame's natural frequency is low, meaning that it provides the same stability as a hand-held weighted gimbal in cinematography. For our application, we'll be using lightweight cameras, which weigh 3 grams each, which means our drone can be lighter.\\

\begin{figure}[H]
\begin{subfigure}{0.5\textwidth}
\centering
\includegraphics[scale=0.11]{images/hex3.jpg}
\caption{Hex multirotor landed}
\end{subfigure}
\begin{subfigure}{0.5\textwidth}
\centering
\includegraphics[scale=0.3]{images/hex2.jpg}
\caption{Hex multirotor in-flight}
\end{subfigure}
\caption{Large cinematography multirotor}
\label{fig:hex}
\end{figure}

Recreational usage may use 'mini multicopter frames' for flying indoors and outdoors \cite{frame} as in Figure \ref{fig:small_quad}. They're extremely light, but do not have enough space or power for other peripherals\footnote{GPS, Raspberry Pis etcetra}.\\

\begin{figure}[H]
\centering
\includegraphics[scale=0.12]{images/small_quad.jpg}
\caption{Small recreational quadcopter}
\label{fig:small_quad}
\end{figure}

Sports drones are light and fast, but they require high discharge batteries. Like the quadcopter\footnote{Four propellers. Hex is 6, and `multirotors' have any number more than 2.} in Figure \ref{fig:small_quad} which has 2300 KV\footnote{KV is not kilo-volts. It is a measure of the revolutions per minute when 1 Volt is applied with no load attached to the motor} rated motors, sports/racing drones also have 2000+ KV ratings, but they will typically have thicker windings, which means it is capable of a higher wattage. One may be able to make enough room, but the batteries will not last very long.\\

A combination of these configurations means that a medium sized quadcopter/drone will be suitable. It is not too light or too heavy, and has room and power for other peripherals.\\

Materials include carbon-fibre, aluminium, fibreglass and synthetic polymers. The differences between them are not major, except that aluminium is heavier, requires larger motors and induces more vibrations. Carbon-fibre contributes to radio interference\cite{frame}, but is the lightest.\\

The closest local-supplier competition before the F450-V2 quadcopter frame was chosen was the ZMR250 Carbon Mini Quad FPV Frame as in Figure \ref{fig:zmr}.

\begin{figure}[H]
\centering
\includegraphics[scale=0.35]{images/zmr250.jpeg}
\caption{ZMR250 Carbon Mini Quad FPV Frame \cite{frobot}}
\label{fig:zmr}
\end{figure}

Both frames have roughly the same price\footnote{500 ZAR}. The ZMR250 has a carbon fibre body making it extremely light (145g), but fell into the class of `mini multicopter' as mentioned earlier. Therefore due to its local availability and accommodating space, the F450-V2 frame was chosen as in Figure \ref{fig:frame}.

\subsubsection{Landing Gear}

Four lengths of 10mm pine dowels approximately 15mm long were used. They totalled about 10 ZAR. The frame has 5cm legs, but accomodation has to be made for the nadir cameras.

\subsubsection{Motors}

The motors are rated at 920 KV and 230 W. This is relatively low compared to the more common racing motors with around 2000 KV or more (which the ZMR250 frame would use). KV is related to the power output and torque level of a motor. This is determined by the number of turns on the armature and the strength of the magnets.\\

\begin{figure}[H]
\centering
\includegraphics[scale=0.4]{images/motor_specs.jpg}
\caption{Motor and propeller combination specifications \cite{frobot}}
\label{fig:mot_prop_specs}
\end{figure}

Besides the many other characteristics, at maximum throttle, and using 3 cell batteries, the thrust is determined to be 3252 g in Figure \ref{fig:mot_prop_specs}. This gives a thrust to weight ratio of 2.95, which is satisfactory.

\subsubsection{Propellers}

The propellers have a pitch of 4.5', which is basically a measure of the 'bite', or distance it travels through the air on one revolution.

\subsubsection{Protective enclosure}

A housing is needed to protect the exposed electronics from the elements. Also, dramatic airflow can affect the barometer readings. A case \cite{3d_case} was 3D printed for the Navio2 and Raspberry Pi as in Figure \ref{fig:fcarpc2}.

\begin{figure}[H]
\begin{subfigure}{0.5\textwidth}
\centering
\includegraphics[scale=0.25]{images/drone-build-3d-case-render.jpg}
\caption{Case render}
\label{fig:fcarpc1}
\end{subfigure}
\begin{subfigure}{0.5\textwidth}
\centering
\includegraphics[scale=0.1]{images/drone-build-3dcase.jpg}
\caption{3D printed case for Raspberry Pi and Navio2.}
\label{fig:fcarpc2}
\end{subfigure}
\caption{Flight controller and Raspberry Pi case}
\label{fig:fcarpc}
\end{figure}

\subsection{Electrical subsystem design}
\subsubsection{Batteries}

Multiple 3S1P batteries will be used as in Figure \ref{fig:batteries}, since the price cost-point was the cheapest from Goblin Hobbies.\\

\begin{figure}[H]
\centering
\includegraphics[scale=0.17]{images/batteries.jpg}
\caption{3SP1 GensAce Lithium batteries}
\label{fig:batteries}
\end{figure}

They are charged using a balance-charger, which charges all three cells equally so that they all age the same. Otherwise cells may burst, or cause damage due to the discrepancies.\\

With an average current draw of 20 A, each battery will only last 6.6 minutes, but long missions can be interrupted easily as noted in Section \ref{sec:dual_power}.

\subsubsection{Dual power redundancy}
\label{sec:dual_power}

The flight controller takes in two power source inputs for dual redundancy as in Figure \ref{fig:dual_redundancy}. One of the batteries powers the motors during flight. This also allows the electronics to remain on when swapping out the flight battery.

\begin{figure}[H]
\centering
\includegraphics[scale=0.35]{images/dual_redundancy.png}
\caption{Illustrating dual power redundancy}
\label{fig:dual_redundancy}
\end{figure}

\subsubsection{Electronic speed controllers}

The 3-phase motors require speed control. This is achieved by ESCs. Maximum power draw from the motors is 18A. 20A rated ESCs are used.

\subsection{Processor subsystem design}

The Navio2 flight controller fits perfectly onto the Raspberry Pi's 40-pin header in Figure \ref{fig:insertion_navio}. It also uses every signal pin, except for one. The Navio2 communicates directly with the Broadcom CPU on the Pi, resulting in a multi-processor system. The greatest significance in this case is that flight variables can be monitored and controlled. It is non-trivial in standalone flight controllers, as the on-board firmware has to be modified with utmost care.\\

The Navio2 has a co-processor to handle PPM/Sbus inputs and provides PWM output for the ESCs.

\subsection{Control subsystem design}

\subsubsection{Flight controller}
Flight controller is needed to stabilize the airborne vehicle, and set mission waypoints. The Raspberry Pi, Navio2, and camera symbiosis was good since all three together are quite configurable even during flight, compared to other solutions which require hands-on intervention.\\

The Navio2 was chosen specifically for its harmonious relationship with the Raspberry Pi.

\subsubsection{Flight modes}

Loiter mode uses GPS to maintain altitude and location. Altitude hold only uses the barometer. Stabilize mode gives the pilot full control of the throttle, and can be used to quickly change height, albeit it is a bit dangerous. Auto mode is used to fly autonomously in missions. Return to launch (RTL) uses GPS and does as its namesake suggests. Brake mode is used to pause its current activity, and if moving quickly it will actively brake by tilting in the other direction to prevent drift. All these flight modes can be accessed in flight by the remote controller, except Auto mode which is activated using the GCS.

\subsection{Communication subsystem design}
\label{sec:comms}

The drone communicates with the handheld remote controller via S-BUS, which is a universal standard. The beauty of it is that it communicates with one signal wire as in Figure \ref{fig:attach_sbus}. Previous implementations one may have had to use pulse position modulation (PPM), where each channel requires a wire. Even though this may sound simple, it does increase PCB size, complexity and cost in the end. One the same note, each electronic speed controller (ESC) gets a signal wire and power input.

%(add picture showing all the wireless technologies)\\

The drone can communicate via wifi as its medium of wireless telemetry; but the interference from other devices in the crowded 2.4 GHz ISM band drastically reduces range -- especially from the handheld remote controller. That, and the wifi dongles that were available seemed to work only for about 10 m.\\

Thus, 433 MHz 100mW transceivers were connected between the GCS and the drone as in Figure \ref{fig:attach_433}, at a 56400 baudrate. Real-time telemtery to a ground station is useful for pre-flight checks, in-flight monitoring and control, and missions.\\

For development purposes, wifi is used extensively to SSH into the Raspberry Pis and to analyse photos using SAMBA. It should be noted that due to the conflict with the remote controller, it is not possible to view photos in-flight. Nevertheless, there are many cases where the wifi disconnects, and it was required to have a script running to reconnect the wifi whenever it goes down (see code in appendix \ref{code:wifiup}).\\

The flight controller controls the ESCs via PWM signals as in Figure \ref{fig:pwm}. Likewise, the remote controller (as in Section \ref{sec:remote_controller}) also sends the PWM values of each channel, yet via SBus (a single-wire form of uart communication). SBus supports up to 18-channels.\\

\begin{figure}[H]
\centering
\includegraphics[scale=0.35]{images/pwm.jpg}
\caption{Min (yellow) and max (blue) duty cycle of PWM}
\label{fig:pwm}
\end{figure}

The PWM has a wavelength of 18 ms, and a duty cycle between 1 ms and 2 ms.

\subsection{Sensor subsystem design}

The flight controller has dual IMUs for redundancy, a GNSS receiver and a high resolution barometer for 10cm altitude resolution.

\begin{enumerate}
\item MPU9250 9DOF IMU
\item LSM9DS1 9DOF IMU
\item MS5611 Barometer
\item U-blox M8N Glonass/GPS/Beidou
\end{enumerate}

\subsection{Construction process and Integration}

At first, the frame is put together. This gives one a good idea of the actual size from the beginning.

\begin{figure}[H]
\begin{subfigure}{0.5\textwidth}
\centering
\includegraphics[scale=0.1]{images/drone-build-frame.jpg}
\caption{F450-V2 frame.}
\label{fig:frame}
\end{subfigure}
\begin{subfigure}{0.5\textwidth}
\centering
\includegraphics[scale=0.1]{images/drone-build-motors.jpg}
\caption{Adding the 920kv motors.}
\label{fig:motors}
\end{subfigure}
\caption{Frame and motors}
\label{fig:frame_motors}
\end{figure}

The nylon polymer frame as in Figure \ref{fig:frame} seems surprisingly robust, especially considering that the centre is PCB based.\\

\begin{figure}[H]
\begin{subfigure}{0.5\textwidth}
\centering
\includegraphics[scale=0.1]{images/drone-build-esc-3phaseunconnected.jpg}
\caption{Adding the ESCs. Motors require 3-phase power}
\label{fig:ESCs_uplugged}
\end{subfigure}
\begin{subfigure}{0.5\textwidth}
\centering
\includegraphics[scale=0.1]{images/drone-build-esc-3phaseconnected.jpg}
\caption{Power leads plugged in and secured}
\label{fig:ESCs_plugged}
\end{subfigure}
\caption{ESCs}
\label{fig:ESC}
\end{figure}

The leads are easy to plug/unplug. Any two phase leads can be swapped to change motor direction as in Figure \ref{fig:ESC}.\\

\begin{figure}[H]
\begin{subfigure}{0.5\textwidth}
\centering
\includegraphics[scale=0.1]{images/drone-build-3dcase-pi.jpg}
\caption{Putting the Pi in the case.}
\label{fig:insertion_pi}
\end{subfigure}
\begin{subfigure}{0.5\textwidth}
\centering
\includegraphics[scale=0.1]{images/drone-build-3dcase-pi-navio.jpg}
\caption{Fitting the Navio2 flight controller on top.}
\label{fig:insertion_navio}
\end{subfigure}
\caption{Inserting the sensitive electronics.}
\label{fig:insertion}
\end{figure}

\begin{figure}[H]
\begin{subfigure}{0.5\textwidth}
\centering
\includegraphics[scale=0.1]{images/drone-build-3dcase-gps.jpg}
\caption{Connecting Ublox Neo-7 GPS antenna and 15-pin camera CSI ribbon cable.}
\label{fig:stab_gps}
\end{subfigure}
\begin{subfigure}{0.5\textwidth}
\centering
\includegraphics[scale=0.1]{images/drone-build-3dplatform.jpg}
\caption{Case and platform}
\label{fig:stab_case_plat}
\end{subfigure}
\caption{Putting the case and platform together}
\label{fig:stabilize_platform}
\end{figure}

The GPS antenna lead fits snugly onto an SMA connector in Figure \ref{fig:stab_gps}, and is exposed in such a way as to leave enough freedom for the cable to bend, but not wear as if it were rigidly attached.

\begin{figure}[H]
\begin{subfigure}{0.5\textwidth}
\centering
\includegraphics[scale=0.1]{images/drone-build-feet.jpg}
\caption{Added feet for platform.} 
\label{fig:feet}
\end{subfigure}
\begin{subfigure}{0.5\textwidth}
\centering
\includegraphics[scale=0.1]{images/drone-build-damper-balls.jpg}
\caption{Rubber vibration damper balls for platform.}
\label{fig:balls}
\end{subfigure}
\caption{Isolating vibrations between flight controller and the rest of the drone}
\label{fig:stabilize_platform}
\end{figure}

One of the biggest problems in a drone is the vibrations emanating from the motors, travelling along the frame and affecting the flight controller. If not isolated from the flight controller, they induce a disturbance to the PID loop since the accuracy of the gyroscope, accelerometer and barometer readings are affected. In some cases, disturbed more than the PID loop can reasonably determine the current state of the drone.\\

Thus, damper balls can be used to isolate vibrations significantly from the flight controller.

\begin{figure}[H]
\begin{subfigure}{0.5\textwidth}
\centering
\includegraphics[scale=0.1]{images/drone-build-case-ondrone.jpg}
\caption{Attaching 3D printed case and platform to drone.}
\label{fig:attach_case_drone}
\end{subfigure}
\begin{subfigure}{0.5\textwidth}
\centering
\includegraphics[scale=0.1]{images/drone-build-433.jpg}
\caption{Adding 433MHz telemtery to drone.}
\label{fig:attach_433}
\end{subfigure}
\caption{Adding telemetry and flight controller}
\label{fig:attach_case_433}
\end{figure}

\begin{figure}[H]
\begin{subfigure}{0.5\textwidth}
\centering
\includegraphics[scale=0.1]{images/drone-build-signal-wires.jpg}
\caption{Wiring up the signal wires}
\label{fig:attach_sbus}
\end{subfigure}
\begin{subfigure}{0.5\textwidth}
\centering
\includegraphics[scale=0.1]{images/drone-build-props.jpg}
\caption{Adding the 9.5'x4.5' propellers}
\label{fig:attach_props}
\end{subfigure}
\caption{Ready to fly}
\label{fig:attach_signal_props}
\end{figure}

\section{Ground Control Station}

The GCS is necessary for pre-flight checks, monitoring/controlling the status of the vehicle during flight, and setting waypoints.\\

The drone communicates with the GCS in Figure \ref{fig:gcs} using 433 MHz transceivers as in Figure \ref{fig:attach_433}.

\begin{figure}[H]
\centering
\includegraphics[scale=0.17]{images/gcs.jpg}
\caption{Ground Control Station}
\label{fig:gcs}
\end{figure}

The software used is the open source Mission Planner platform.

\begin{figure}[H]
\centering
\includegraphics[scale=0.2]{images/mp.jpg}
\caption{Mission planner}
\label{fig:mission_planner}
\end{figure}

\section{Remote controller}
\label{sec:remote_controller}

The remote controller is used to manually control the drone. It is also used to monitor the battery and signal strength.\\

\begin{figure}[H]
\begin{subfigure}{0.5\textwidth}
\centering
\includegraphics[scale=0.255]{images/taranis.png}
\caption{Taranis Q X7 remote controller}
\label{fig:remote_controller}
\end{subfigure}
\begin{subfigure}{0.5\textwidth}
\centering
\includegraphics[scale=0.15]{images/frsky_transceiver.jpg}
\caption{FrSky X4R transceiver}
\label{fig:balls}
\end{subfigure}
\caption{Remote controller and on-board transceiver}
\label{fig:frsky_transceiver}
\end{figure}

The full-duplex remote controller outputs at 100mW. The FrSky X4R transceiver has two antennae, and also outputs at 100 mW.\\

At first, the Taranis remote controller is bound to the FrSky X4R transceiver aboard the drone in Figure \ref{fig:frsky_transceiver}. This is done by holding in the `F\/S' button on the transceiver while powering it on, and entering bind mode on the remote controller.

\begin{figure}[H]
\begin{subfigure}{0.5\textwidth}
\centering
\includegraphics[scale=0.13]{images/lcd/flight_mode_but_1.jpg}
\caption{Linearised flight mode switch 1 input}
\label{fig:taranis_fm_but1}
\end{subfigure}
\begin{subfigure}{0.5\textwidth}
\centering
\includegraphics[scale=0.13]{images/lcd/flight_mode_but_2.jpg}
\caption{Linearised flight mode switch 2 input}
\label{fig:taranis_fm_but2}
\end{subfigure}
\begin{subfigure}{0.5\textwidth}
\centering
\includegraphics[scale=0.13]{images/lcd/mixer.jpg}
\caption{Mixer muxes input channels to output}
\label{fig:taranis_mux}
\end{subfigure}
\begin{subfigure}{0.5\textwidth}
\centering
\includegraphics[scale=0.13]{images/lcd/telemetry.jpg}
\caption{Remote sensor feedback}
\label{fig:taranis_telemetry}
\end{subfigure}
\caption{Some important settings for the remote controller}
\label{fig:taranis_lcd}
\end{figure}

The two 3-way flight mode input switches as indicated in Figure \ref{fig:taranis_arm} are multiplexed into channel 5 as in Figure \ref{fig:taranis_mux}. As indicated in Figure \ref{fig:taranis_fm_but1} and \ref{fig:taranis_fm_but2}, the two switches required a peculiar linearisation such that when they are multiplexed, there exist 6 PWM positions within the duty cycle range. \begin{equation}PWM\ flight\ mode\ output\ values\ \in [1165, 1295, 1425, 1555, 1685, 1815]\ ms\end{equation}
This allows the PWM outputs to exist within the ranges as indicated in Figure \ref{fig:flight_modes}.

\begin{figure}[H]
\centering
\includegraphics[scale=0.5]{images/flight_modes.JPG}
\caption{Flight modes}
\label{fig:flight_modes}
\end{figure}

To arm the drone for takeoff, move the left controller stick to the bottom right for 2 seconds, as in Figure \ref{fig:taranis_arm}. Also, lift the safety `Cut motors' switch.

\begin{figure}[H]
\centering
\includegraphics[scale=0.17]{images/taranis_arm.jpg}
\caption{Arming the drone (red arrow)}
\label{fig:taranis_arm}
\end{figure}

\section{Software}

This will be handled by two Raspberry Pis. Both will be used to take photos through each of their single CSI ports. The Raspberry Pis will communicate with each other via direct ethernet connection. Twisted-pair is not necessary since the driver automatically swaps the RX and TX lines.\\

Initially, one Raspberry Pi was to be used with a camera multiplexer as in Section \ref{sec:simultaneous_trig}, but it didn't work, and therefore two raspberry pis were used. Just as well, as the flight controller uses all but 1 GPIO pin.\\

A real-time kernel will be used for the Ardupilot Firmware used on the `flight controller' Raspberry Pi, extended by a Navio2. \\

\section{Testing}

There was a case where one foot holding vibration dampeners broke and induced an unstable asynchronous frequency when rolling in a certain direction as in Figure \ref{fig:leafy}.

\begin{figure}[H]
\centering
\includegraphics[scale=0.58]{images/pitch_alt.png}
\caption{Oscillations in pitch and roll}
\label{fig:leafy}
\end{figure}

The PID values for pitch and roll rate were lowered by about 11\% to account for the frame size as in Figure \ref{fig:tuning}.\\

\begin{figure}[H]
\centering
\includegraphics[scale=0.5]{images/tuning.PNG}
\caption{Tuning PID values}
\label{fig:tuning}
\end{figure}

%Relevant flight log data.\\

An RTL2832U software defined radio (SDR) with a receive sensitivity of -140dBm was used to test the range of the 433 MHz transceivers.

\begin{figure}[H]
\begin{subfigure}{0.5\textwidth}
\centering
\includegraphics[scale=0.45]{images/range_test.PNG}
\caption{FHSS channel hopping}
\label{fig:range_test}
\end{subfigure}
\begin{subfigure}{0.5\textwidth}
\centering
\includegraphics[scale=0.8]{images/range_test_700m.PNG}
\caption{700m NLOS before lost signal}
\label{fig:range_test_700m}
\end{subfigure}
\caption{Range testing}
\label{fig:range}
\end{figure}

Also, compass interference was measured against current draw as in Figure \ref{fig:interference} to lower the chance of errors.

\begin{figure}[H]
\centering
\includegraphics[scale=0.35]{images/comp_mot_true_current.PNG}
\caption{Measuring compass interference vs current draw.}
\label{fig:interference}
\end{figure}


%Many other tests were performed on each component subsystem. For example, a current meter was used to verify the current constants for both power modules, and 
\chapter{Image Acquisition and Calibration}

Since cross-talk between channels cannot be eliminated completely, there needs to be two cameras due to the Bayer matrix colour array (see section \ref{sec:intrinsic}). Attempts to have both on a single camera are ineffective (explain why).\\

The following concepts are crucial to understand the reasoning behind certain decisions.\\

In consumer cameras, the filter pattern is arranged in an RGGB format as in Figure \ref{fig:bayer}, to mimic the physiology of the eye. Demosiacing is performed to interpolate the neighbouring pixel colours for every pixel.

\begin{figure}[H]
\centering
\includegraphics[scale=0.35]{images/bayer.png}
\caption{Bayer arrangement of colour filters on an image sensor's pixel array \cite{bayer}}
\label{fig:bayer}
\end{figure}

The colour filters do allow infra-red light in as well, which is why consumer cameras generally have a NIR longpass filter in place to isolate RGB colours.\\

Due to variances in focal length, and slight manufacturing disparities, there is some rotation and translation in real lenses.\\

There is also distortion, as in Figure \ref{fig:distortion_examples}.

\begin{figure}[H]
\centering
\includegraphics[scale=0.5]{images/distortion_examples.png}
\caption{Distortion examples \cite{calib3d}}
\label{fig:distortion_examples}
\end{figure}

This can be modelled by the pinhole camera model as in section \ref{sec:pinhole}.

%\subsection{Camera Calibration}
%
%(picture of undistortion).
%
%\subsubsection{Jello effect}
%
%ripple


\section{Filters}

A filter is needed to isolate the NIR channel. If a custom filter is used on a camera with the NIR filter removed, it is possible to isolate a NIR channel without visible light leakage. A few of the variants include the Wratten 25A Red Long Pass filter, the Wratten 87 NIR All Pass filter and the Rosco 2008 Blue filter, which allow red and NIR light, only NIR light, and blue and NIR light respectively.\\

The Rosco 2008 Blue filter will be used, for most of the project, as well as the 600nm Dichroic Glass Longpass filter and 500nm Dichroic Glass Bandpass filter from Edmund Scientific, with limited access.\\

The blue Rosco 2008 filter is used as in Figure \ref{fig:blue_filter}, and passes only NIR within the green and red channels, as shown in Figure \ref{fig:blue_curve}.

\begin{figure}[H]
\begin{subfigure}{0.5\textwidth}
\centering
\includegraphics[scale=0.45]{images/blue_filter.jpg}
\caption{Blue Rosco 2008 filter \cite{blue_filter}}
\label{fig:blue_filter}
\end{subfigure}
\begin{subfigure}{0.5\textwidth}
\centering
\includegraphics[scale=0.42]{images/superblueinfraredfiltercurve.png}
\caption{Blue filter transmission curve.}
\label{fig:blue_curve}
\end{subfigure}
\caption{Blue filter characteristics \cite{blue_curve}}
\label{fig:blue_character}
\end{figure}

%\begin{figure}[H]
%\centering
%	
%\subfloat[Blue Rosco 2008 filter \cite{blue_filter}]
%	\includegraphics[scale=0.45, width=0.5\textwidth]{images/blue_filter.jpg}
%	\label{fig:blue_filter2}
%
%	
%\subfloat[Blue filter transmission curve.]
%	\includegraphics[scale=0.42]{images/superblueinfraredfiltercurve.png}
%	\label{fig:blue_curve2}
%
%\caption{Blue filter characteristics \cite{blue_curve}}
%\label{fig:blue_characte2r}
%\end{figure}
%
%\section{Cameras}

Two cameras are required. It is seemingly impossible to capture pure NIR and pure red on separate channels unless the intrinsic bayer matrix is manufactured to allow for it, however, then it would not be cost effective (citation needed).\\

The Sony IMX219 series cameras will be used. They integrate with the Raspberry Pi using single 15-pin CSI connections.

\section{Exposure}

It is important that the correct white-balance settings are chosen so as not to allow extremeties such as in Figure \ref{fig:exposure} to occur.

\begin{figure}[H]
\begin{subfigure}{0.5\textwidth}
\centering
\includegraphics[scale=0.17]{images/under-exposed.jpg}
\caption{Under exposed}
\label{fig:under_exposed}
\end{subfigure}
\begin{subfigure}{0.5\textwidth}
\centering
\includegraphics[scale=0.17]{images/over-exposed.jpg}
\caption{Over exposed}
\label{fig:over_exposed}
\end{subfigure}
\caption{Incorrect exposure settings}
\label{fig:exposure}
\end{figure}

A calibration plate (see section x) should resolve drift in NDVI values caused by clouds or sunshine (glare might be more of an issue).

\section{Camera mount}

The motors cause high-frequency vibrations, and if not removed, it can cause what is known as the Jello-effect as in Figure \ref{fig:jello_effect}.

\begin{figure}[H]
\centering
\includegraphics[scale=0.27]{images/ripple.jpg}
\caption{Jello-effect visible before dampening}
\label{fig:jello_effect}
\end{figure}

The cameras will be fixed relative to each other using a PCB substrate so that no rotational or translational drift can occur as in Figure \ref{fig:cam_mount}.\\

Also, the cameras are to be dampened as in Figure \ref{fig:dampener} to prevent a jello-effect due to the rolling shutter cameras. Global shutter cameras are more expensive (cite).

\begin{figure}[H]
\begin{subfigure}{0.5\textwidth}
\centering
\includegraphics[scale=0.3]{images/dampener.jpg}
\caption{Dampening for the cameras}
\label{fig:dampener}
\end{subfigure}
\begin{subfigure}{0.5\textwidth}
\centering
\includegraphics[scale=0.3]{images/fixed_orientation.jpg}
\caption{Fixed orientation for cameras}
\label{fig:fixed_orientation}
\end{subfigure}
\caption{Camera mount}
\label{fig:cam_mount}
\end{figure}

Although a DC brushless motorised gimbal would aid in nadir orientation, increasing the overlap between images should account for when the cameras are not pointing directly downwards.

\section{Simultaneous triggering}

Initially, an IVmech Camaera multiplexer was used as in Figure \ref{fig:ivmech}. Unfortunately, it didn't work. Upon debugging using a logic analyser, the $CAM\_GPIO$ and $CAM1\_DN$ pins were not multiplexed as they should have been.

\begin{figure}[H]
\centering
\includegraphics[scale=0.25]{images/ivmech.jpg}
\caption{IVmech camera multiplexer}
\label{fig:ivmech}
\end{figure}

With the dual raspberry pi setup, TCP triggering was used as in code appendix \ref{code:tcp_trigger}, but the overhead and asynchronous nature meant that it was never consistently triggered at the same time. Instead, the clocks are sychronised using NTP, and attempt to trigger every second, on the second. It attempts to, because sometimes the GPU runs garbage collection on its memory to make way for new images. Except for the garbage collection period (which takes a few seconds), the latter solution works quite well, with an occassional few milliseconds difference in stereo captures. This is accounted for in Chapter \ref{sec:image_processing} using feature matching and homography.

\section{The Camera Model}
\subsection{Camera Geometry}
\label{sec:pinhole}

It is important to be able to model the cameras in a mathematical form so that variances can be accounted for in each unique camera.\\

The cameras can be approximated as pin-hole camera models with parameters as in section \ref{sec:pinhole}. Estimating the parameters is known as 'camera resectioning'. Light enters camera through focal point $Fc$, and falls onto the sensor, as depicted in Figure \ref{fig:camera_geometry}. $f$ is the focal length, which is the distance between the image sensor and lens. The symmetry of the model means that the correct orientation is shown at the image plane.

\begin{figure}[H]
\centering
\includegraphics[scale=0.35]{images/camera_geometry.png}
\caption{Geometry of a single camera}
\label{fig:camera_geometry}
\end{figure}

The field of view is restricted by sensor size and focal length, and depends on the distance (or height) of the camera from the subject. Since a stereo setup is used, the field of view overlaps, but not from the same focal point as shown in Figure \ref{fig:stereo_overlap}. The idea would be to superimpose the two image planes one top of each other.

\begin{figure}[H]
\centering
\includegraphics[scale=0.45]{images/stero_overlap.jpg}
\caption{Stereo vision overlap}
\label{fig:stereo_overlap}
\end{figure}

%various coordinate systems?\\
%Cartesian?\\
%right-hand rule?\\
%multiple-cameras used, therefore epiline geometery?\\
%relationship between coordinate systems, camera, image plane, system environment.\\\\
%
%width Xc\\
%height Yc\\
%depth Zc\\
%Focal point Fc set at origin\\
%P exists within R\^3\\\\
%
%image plane provides two coordinate systems, the u-v coordinate system, and the x-y coordinates.\\
%H, W\\
%position of the image plane in conjunction to the cam geometry\\\\
%
%the point at which the principle axis Zc penetrates the image plane is referred to as the principle point and denoted as (Cx, Cy). The principle point is set as the origin of the x-y coordinate system. The point P is bounded within the image plane as so:\\
%Pu, Pv, Px, Py, W, H\\\\
%
%The 3d env defining fields = object space. show origin space relative to other spaces. Focal poitn Fc is aligned\\
%with origin of obj space such that\\
%Zc == Z\\
%obj plane is parallel to the image plane and describes the range of possible X-Y positions of the object at a set distance.\\
%The distance between the focal point and the object plane is workign distance,\\
%denoted as Zd. and can be used to determione the feild of view.\\\\
%
%\subsection{The Camera Model}
%
%camerai is an optical instrument used to capture and store visual info.\\
%the cam model is a mathematical representation of the internal camera geometry\\
%and can be extended to define the position of the camera within the object space.\\
%the camera model is an essential tool used to construct the relation between the camera, image plane and obj space.\\\\
%
%Calibration depends on the info given by the camera model to effectively rectify differences in the stereo setup.\\
%This chapter therefore investigates the construction of the camera model, alogn with the coresponding paramters, to establish how the 3-d object space will be portrayed ion the 2d sapce.\\
%
%The camera model is composed from variuos parameters which are initially unkown The process of identifying or calculating these unkonw params is known as camera calibration.


\subsection{Intrinsic Parameters}
\label{sec:intrinsic}

The internal camera geometry describes the relation between the camera and the image plane. This relation is determined by the focal length, principle point and possible axis misalignment. These defining parameters are referred to as the intrinsic parameters and are measured in pixels.\\

The intrinsic parameters are often represented as a matrix known as the camera matrix. The camera matrix $M_c$ for each camera is defined as:

\begin{equation}\label{eq:cm}
M_c^{(j)}\,=\,\begin{bmatrix}
{f_x^{(j)}} & {s} & {c_x^{(j)}}\\
{0} & {f_y^{(j)}} & {c_y^{(j)}}\\
{0} & {0} & {1}
\end{bmatrix},\quad j = 1,\, 2
\end{equation}

where:\\
$f$ is the focal length\\
$s$ is the skew,\\
and $Cx, Cy$ is the x and y coordinate of the principle point.

\subsection{Extrinsic Parameters}

The extrinsic parameters describe the correspondence between the two camera coordinate systems. The extrinsic parameters include rotation matrix $R$ and translation matrix $T$ to describe how the one coordinate system will map to the other.\\

These intrinsic and extrinsic parameters are used to construct the pin hole camera model.

\subsection{Pinhole Camera model}
\label{sec:pinhole}

The pinhole camera model describes the ideal transformation between a point within the object space and the representation of that point in the image plane. 

%Section 2.3 (cam geometry) provides the description of the pinhole camera concept. 

\begin{figure}[H]
\centering
\includegraphics[scale=0.75]{images/pinhole_model.JPG}
\caption{Pinhole camera model \cite{calib3d}}
\label{fig:pinhole_model}	
\end{figure}

The point P is found in figure \ref{fig:pinhole_model} and is defined within the 3d object space as:

\begin{equation}\label{eq:principleP}
P =  \begin{bmatrix}{X}\\{Y}\\{Z}\end{bmatrix}
\end{equation}

The extrinsic parameters are used to describe P in terms of the (x, y, z) coordinate system illustrated in blue in figure.

\begin{equation}\label{eq:principleP_transformation}
\begin{bmatrix}{x}\\{y}\\{z}\end{bmatrix} = R  \begin{bmatrix}{X}\\{Y}\\{Z}\end{bmatrix}\,+\,T\\
\end{equation}

where:\\
$R$ is the rotational matrix\\
$T$ is the translational vector\\

If the extrinsic parameters are known, then:

\begin{align*}
R = I, and\\
T = 0
\end{align*}

and equation \ref{eq:principleP_transformation} is altered as follows 

\begin{equation}\label{altered}
\begin{bmatrix}{x}\\{y}\\{z}\end{bmatrix} =  \begin{bmatrix}{X}\\{Y}\\{Z}\end{bmatrix}
\end{equation}\\

If the distance $z$ is equal to the focal length $f$, the x-y will represent the 2-D image plane. The representation of P remains unchanged if the the coordinates are multiplied by a common factor. This multiplication conceptually shifts the x-y plane along the principle axis. The representation of point $P$ in equation \ref{altered} is consequently modified as follows:

\begin{equation}x' = x/z,\quad and \end{equation}
\begin{equation}y' = y/z \end{equation}

under the condition that \begin{align*} z \neq 0 \end{align*}

The image plane will therefore represent $P$ in terms of the $u-v$ coordinate system, defined in figure x as 

\begin{equation}\label{eq:uv1}u = f_x*x'' + c_x \\ \end{equation}
\begin{equation}\label{eq:uv2}v = f_y*y'' + c_y \end{equation}

where:\\
$(c_x, c_y)$ is the principle point, and\\
$f$ is the focal length.\\

Equation \ref{eq:uv1} and \ref{eq:uv2} can also be written as:

\begin{equation}
\begin{bmatrix}u\\v\\l\end{bmatrix} =  
\begin{bmatrix}
f & 0 & C_x\\
0 & f & C_y\\
0 & 0 & 1
\end{bmatrix}
\begin{bmatrix}x'\\y'\\1\end{bmatrix}
\end{equation}

The derivation of the camera matrix $M_c$ in equation \ref{eq:cm} is effectively as performed above. The pinhole camera model assumes ideal alignment and therefore no skew is present within the model. This concludes the construction of the camera model.

\section{Camera Stereo-calibration}

Camera calibration can now be performed to identify the unknown parameters within the camera model.

\subsection{Calibration technique}
\label{sec:cal_technique}

The calibration technique provided by OpenCV was used to identify the unknown parameters. The tutorial adopts the corner detection calibration method described below.\\

The corner detection calibration method is a common calibration technique due to the simplicity of the calibration process.  A chessboard is used as a reference point for the calibration even if the dimensions are unknown, it can be assume that the squares exist on a 2-D plane in a 3-D space. Multiple images containing various orientations of the chessboard are used to detect the size and orientation of the squares. These measurements are then used to determine the unknown parameters. Figure \ref{fig:rgb_input_chess} and \ref{fig:ir_input_chess} shows how the chessboard is observed by the camera for multiple orientations. 

\begin{figure}[H]
\centering
\includegraphics[scale=0.505]{images/rgb_chess_set.JPG}
\caption{RGB input chess images}
\label{fig:rgb_input_chess}
\end{figure}

\begin{figure}[H]
\centering
\includegraphics[scale=0.5]{images/ir_chess_set.JPG}
\caption{NIR input chess images}
\label{fig:ir_input_chess}
\end{figure}

Stereo-calibration using 16x16 square chess board images were used.\\

Equations\\

(note: still to be expanded upon)\\

\begin{equation}\label{eq:dist}
dist\,=\,\begin{bmatrix}
k_1 & k_2 & p_1 & p_2 & k_3
\end{bmatrix}
\end{equation}

\begin{equation}\label{eq:1} R_2=R*R_1T_2=R*T_1 + T \end{equation}

\begin{equation}\label{eq:2}
E\,=\,\begin{bmatrix}
{0} & {-T_2} & {T_1}\\
{T_2} & {0} & {-T_0}\\
{-T_1} & {T_0} & {0} 
\end{bmatrix}\,*\,R
\end{equation}

\begin{align*}
F = cameraMatrix_2^{-T}\cdot E\cdot cameraMatrix_1^{-1} 
\end{align*}

\begin{align*}
T : T=[T_0, T_1, T_2]^T  
\end{align*}

\begin{equation}\label{eq:P1}
P1 = \begin{bmatrix} f & 0 & cx_1 & 0 \\ 0 & f & cy & 0 \\ 0 & 0 & 1 & 0 \end{bmatrix}
\end{equation}

\begin{equation}\label{eq:P2}
P2 = \begin{bmatrix} f & 0 & cx_2 & T_x*f \\ 0 & f & cy & 0 \\ 0 & 0 & 1 & 0 \end{bmatrix}
\end{equation}

%\begin{equation}\label{eq:3}
\begin{equation}x'' = x'  \frac{1 + k_1 r^2 + k_2 r^4 + k_3 r^6}{1 + k_4 r^2 + k_5 r^4 + k_6 r^6} + 2 p_1 x' y' + p_2(r^2 + 2 x'^2)  \\ \end{equation}
\begin{equation}y'' = y'  \frac{1 + k_1 r^2 + k_2 r^4 + k_3 r^6}{1 + k_4 r^2 + k_5 r^4 + k_6 r^6} + p_1 (r^2 + 2 y'^2) + 2 p_2 x' y'  \\
\quad \text{where} \quad r^2 = x'^2 + y'^2  \\ \end{equation}

\subsection{Measured Results}

(note: still to be expanded upon)

\begin{equation}\label{eq::cm1}
M_c^1 = \begin{bmatrix}
2.523278 & 0 & 1.64 \\
0 & 2.516529 & 1.232 \\
0 & 0 & 0.001
\end{bmatrix}\cdot10^3
\end{equation}

\begin{equation}\label{eq::cm2}
M_c^2 = \begin{bmatrix}
2.505381 & 0 & 1.64 \\
0 & 2.505083 & 1.232 \\
0 & 0 & 0.001
\end{bmatrix}\cdot10^3
\end{equation}

\begin{equation}\label{eq::R}
R = \begin{bmatrix}
999.281 & -6.4992 & 37.35413 \\
6.59547 & 999.97524 & -2.45456 \\
-37.33725 & 2.69917 & 999.29908
\end{bmatrix}\cdot10^3
\end{equation}

\begin{equation}\label{eq::E}
E = \begin{bmatrix}
-0.3150989 & 158.311619 & 35.973848 \\
-55.293721 & -6.403995 & -2757.525727 \\
-18.200605 & 2753.713987 & -8.117986
\end{bmatrix}\cdot10^-3
\end{equation}

\begin{equation}\label{eq::F}
F = \begin{bmatrix}
-2.54558709\cdot10^-6 & 1.28238104\cdot10^-3 & -0.842399343\\
-0.446754\cdot10^-3 & -51.880818\cdot10^-6 & -55.4216639\\
0.1861918 & 53.846 & 1000
\end{bmatrix}\cdot10^-3
\end{equation}

\begin{equation}\label{eq::R1}
R_1 = \begin{bmatrix}
0.99977728 & 0.00654886 & -0.02006268\\
0.00657528 & 0.9999776 & -0.00125126\\
0.02005403 & 0.0013829 & 0.99979794
\end{bmatrix}
\end{equation}

\begin{equation}\label{eq::R2}
R_2 = \begin{bmatrix}
0.99826641 & 0.01319195 & -0.05735987\\
-0.01311635 & 0.99991254 & 0.00169423\\
0.05737721 & -0.00093894 & 0.99835213
\end{bmatrix}
\end{equation}

\begin{equation}\label{eq::P1}
P_1 = \begin{bmatrix}
2505.083324 & 0 & 1778.704819 & 0\\
0 & 2505.083324 & 1231.092529  & 0\\
0 & 0 & 1 & 0
\end{bmatrix}
\end{equation}

\begin{equation}\label{eq::P2}
P_2 = \begin{bmatrix}
2505.083324 &    0.       & 1778.704819 & 6909.840227\\
 0          & 2505.083324 & 1231.092529 &    0        \\
 0          &    0        &    1        &    0         \\
\end{bmatrix}
\end{equation}

\begin{equation}\label{eq::Q}
Q\ =\ \begin{bmatrix}
1 & 0 & 0 & -1.778705\cdot10^3\\
0 & 1 & 0 & -1.231093\cdot10^3\\
0 & 0 & 0 & 2.505083\cdot10^3\\
0 & 0 & -0.362539 & 0
\end{bmatrix}
\end{equation}

\begin{equation}\label{eq::T}
T\ =\ \begin{bmatrix}
2.75354568\\
0.03638771\\
-0.15821732
\end{bmatrix}
\end{equation}

\begin{equation}\label{eq::validpix1}
valid\ Pix\ ROI_1\ =\ 
\begin{bmatrix}
x_1 & y_1\\
x_2 & y_2
\end{bmatrix}\ =
\begin{bmatrix}
88 & 6\\
3192 & 2421
\end{bmatrix}
\end{equation}

\begin{equation}\label{eq::validpix2}
valid\ Pix\ ROI_2\ =\ 
\begin{bmatrix}
x_1 & y_1\\
x_2 & y_2
\end{bmatrix}\ =
\begin{bmatrix}
0 & 26\\
3204 & 2378
\end{bmatrix}
\end{equation}

\begin{equation}\label{eq::error1}
error_1 = 17.255\%
\end{equation}
\begin{equation}\label{eq::error2}
error_2 = 14.647\%
\end{equation}

\section{Stereo-rectification}

The images were undistorted (see chapter 4 on intrinsic parameters) according to the simultaneous chess board images it was calibrated at.

\subsection{Measured Results}

\section{Undistortion}
\label{sec:undistortion}

Undistortion of the images is required, as in Figure \ref{fig:distortion_examples}.

\begin{figure}[H]
\centering
\includegraphics[scale=0.25]{images/incorrectly_undistorted.jpg}
\caption{Undistortion when using 8x8 chessboard image set with images not close enough to the edges}
\label{fig:incorrectly_undistorted}
\end{figure}

In Figure \ref{fig:undistorted_roi}, the 16x16 chess set was used for stereo-calibration.

\begin{figure}[H]
\begin{subfigure}{0.5\textwidth}
\centering
\includegraphics[scale=0.17]{images/rgb_undistorted.jpg}
\caption{RGB image}
\label{fig:rgb_undistorted}
\end{subfigure}
\begin{subfigure}{0.5\textwidth}
\centering
\includegraphics[scale=0.17]{images/ir_undistorted.jpg}
\caption{Corresponding IR image}
\label{fig:ir_undistorted}
\end{subfigure}
\caption{Correctly undistorted images with valid ROI outlined}
\label{fig:undistorted_roi}
\end{figure}
\chapter{Image Processing and NDVI}
\label{sec:image_processing}
SIFT, mapping, geotagging, alignment,\\

Once the images are undistorted as in Section \ref{sec:undistortion}, the images can be aligned using stereo-correspondence before an NDVI is performed.\\

A scale-invariant-feature-transform is applied to the images to align them using an affine transform. An NDVI calculation is applied to the matched images, with a floating point output image. A LUT colourmap is applied to visibly distinguish different areas and their meanings in the final image.

\section{Feature Detection}

There are different types of feature detection. 
\begin{itemize}
	\item Harris Corner Detection
	\item Shi-Tomasi Corner Detection
	\item SIFT (Scale-Invariant Feature Transform)
	\item SURF (Speeded-Up Robust Features)
	\item FAST (Features from Accelerated Segment Test)
	\item BRIEF (Binary Robust Independent Elementary Features)
	\item ORB (Oriented FAST and Rotated BRIEF)
	\item BRISK, FREAK, KAZE, and AKAZE
\end{itemize}

There are also different types of feature matching.

\begin{itemize}
	\item Feature Matching + Homography to find Objects
	\item FLANN (Fast Library for Approximate Nearest Neighbors) based matcher
\end{itemize}

Harris Corner detection is useful for the chessboard calibration technique as in Section \ref{sec:cal_technique}. Shi-Tomasi improved the scoring function of Harris, but it is more appropriate for tracking. SIFT provides keypoints and descriptors. It also does not depend on the scale of a corner, unlike in Figure \ref{fig:sift_scale}. SURF is good at dealing with blurring and rotation, but not at handling viewpoint and illumination change as in SURF. FAST is better from a realtime, limited resource application point of view. Although it is several times faster than the other detectors, it is not robust against high levels of noise, and depends on a threshold. BRIEF is a quicker feature descriptor with lower memory usage, but feature detection such as SIFT, SURF is still needed. ORB is a good alternative to SIFT and SURF in performance, and is not patented.\\

\begin{figure}[H]
\centering
\includegraphics[scale=0.5]{images/sift_scale_invariant.jpg}
\caption{Harris corner detection on scaled corners. \cite{calib3d}}
\label{fig:sift_scale}
\end{figure}

\subsection{Feature matching}

SIFT returns correspondence points between the stereo images.\\


It will then use brute-force feature matching, which calculates for each descriptor in the first image, the distance between it and all the features in the second image, returning the closest one. If the second closest corresponding point is within a threshold normalized distance of 0.8, according to Louw, it is discarded. The remaining candidates are filtered by geometric consensus.\\

Once the corresponding features are found, an affine transform matrix is found. For example,

%SURF is faster than SIFT, since it approximates the Laplacian of Gaussian with Box Filter instead of Difference of Gaussian.

\subsection{Homography}



\section{Stitching and mapping}

A few images are stitched together to show proof-of-concept as in Figure \ref{fig:stitch_map}.

\begin{figure}[H]
\centering
\includegraphics[scale=0.35]{images/ndvi_stitch_example.jpg}
\caption{Photos stitched into a map}
\label{fig:stitch_map}
\end{figure}

\section{Calibration Plate}
\label{sec:cal_plate}

\begin{figure}[H]
\begin{subfigure}{0.5\textwidth}
\centering
\includegraphics[scale=0.17]{images/rgb_cal.jpg}
\caption{RGB image}
\end{subfigure}
\begin{subfigure}{0.5\textwidth}
\centering
\includegraphics[scale=0.17]{images/ndvi_cal.jpg}
\caption{Processed NDVI image}
\end{subfigure}
\caption{Calibration plate}
\label{fig:cal_plate}
\end{figure}
\chapter{System Integration and Testing}

quick discussion, issues

testing, + results, + analysis

Capabilities, works well, could be better, compared to this solution. Parrot Sequoia. Quite affordable. Complexity?

\chapter{Conclusion}

Overall, the system seems capable. It is much cheaper than equivalent products on the market today, and would require some refining to be robust, and consistently useful for farmers.\\

The drone design and construction, image acquisition, stereo rectification and calibration was a success, but the mapping and NDVI calibration will need some further investigation.

% In order for the data to be transmitted successfully, it needs to be in a format acceptable by the GCS. The data will be stored in a JSON packet created using a C++ library\cite{ajson}.

% Code block \ref{code:json1} shows a potential data packet.

% \lstset{language=html,caption={Potential packet},label=code:json1}
% \begin{lstlisting}
% {
%   "temperature": "11.3",
%   "pressure": "99.325"
% }
% \end{lstlisting}

% \begin{figure}
% \centering
% \includegraphics[scale=0.35]{flight_path.png}
% \caption{CanSat flight trajectory\\
% Reproduced from \cite{gopher}}
% \label{fig:flight_path}
% \end{figure}




% \subsection{How to add Tables}

% Use the table and tabular commands for basic tables --- see Table~\ref{tab:widgets}, for example. 

% \begin{table}
% \centering
% \begin{tabular}{l|r}
% Item & Quantity \\\hline
% Widgets & 42 \\
% Gadgets & 13
% \end{tabular}
% \caption{\label{tab:widgets}An example table.}
% \end{table}

% \subsection{How to write Mathematics}

% \LaTeX{} is great at typesetting mathematics. Let $X_1, X_2, \ldots, X_n$ be a sequence of independent and identically distributed random variables with $\text{E}[X_i] = \mu$ and $\text{Var}[X_i] = \sigma^2 < \infty$, and let
% \[S_n = \frac{X_1 + X_2 + \cdots + X_n}{n}
%       = \frac{1}{n}\sum_{i}^{n} X_i\]
% denote their mean. Then as $n$ approaches infinity, the random variables $\sqrt{n}(S_n - \mu)$ converge in distribution to a normal $\mathcal{N}(0, \sigma^2)$.


% \subsection{How to create Sections and Subsections}

% Use section and subsections to organize your document. Simply use the section and subsection buttons in the toolbar to create them, and we'll handle all the formatting and numbering automatically.

% \subsection{How to add Lists}

% You can make lists with automatic numbering \dots

% \begin{enumerate}
% \item Like this,
% \item and like this.
% \end{enumerate}
% \dots or bullet points \dots
% \begin{itemize}
% \item Like this,
% \item and like this.
% \end{itemize}

% \subsection{How to add Citations and a References List}

% You can upload a \verb|.bib| file containing your BibTeX entries, created with JabRef; or import your \href{https://www.overleaf.com/blog/184}{Mendeley}, CiteULike or Zotero library as a \verb|.bib| file. You can then cite entries from it, like this: \cite{greenwade93}. Just remember to specify a bibliography style, as well as the filename of the \verb|.bib|.

% You can find a \href{https://www.overleaf.com/help/97-how-to-include-a-bibliography-using-bibtex}{video tutorial here} to learn more about BibTeX.

% We hope you find Overleaf useful, and please let us know if you have any feedback using the help menu above --- or use the contact form at \url{https://www.overleaf.com/contact}!

\begin{appendices}

More info can be found here: https://github.com/daniel-leonard-robinson/thesis

%\chapter{Bill of Materials}
%\label{app:bom}
%
%\begin{itemize}
%	\item F450 frame
%	\item Motors
%	\item 
%\end{itemize}

\chapter{Moisture sensor}

A cheap moisture sensor from China was investigated as in Figure \ref{fig:moisture_sensor}. It might be useful if it can be used for ground-truthing.

\begin{figure}[H]
\centering
\includegraphics[scale=0.17]{images/moisture_sensor.jpg}
\caption{Moisture sensor}
\label{fig:moisture_sensor}
\end{figure}

The sensor measures the resistivity of the soil, and values range from about 5V for an open circuit, and about 1V for different types of water (salty, brackish etc). A typical soil value would be about 1.5V for wet soil, and 4V for dry soil. Unfortunately, there was too much variance for the same type of soil, and it was determined that more accurate equipment is needed. It is also more expensive, but worth investigating.

\chapter{Spectrum analyzer}

A spectrum analyzer kit was built to investigate what can be found from diffraction and perhaps identifying spectral characteristics of cameras.

\begin{figure}[H]
\centering
\includegraphics[scale=0.4]{images/collimeter.jpg}
\caption{Acetate collimeter}
\label{fig:coll}
\end{figure}

\begin{figure}[H]
\centering
\includegraphics[scale=0.4]{images/spectrum.jpg}
\caption{Diffraction from light source}
\label{fig:diffraction}
\end{figure}

Unfortunately it proved too flimsy, and that laboratory equipment would be better suited.


\chapter{Filter comparison}
\label{app:filter_comparison}

See filters in Section \ref{sec:filters} for an in-depth discussion.

\begin{figure}[H]
\centering
\includegraphics[scale=0.2]{filter/rgb.jpg}
\caption{RGB image overlooking vegetation outside window}
\label{fig:f_rgb}
\end{figure}

\begin{figure}[H]
\begin{subfigure}{0.5\textwidth}
\centering
\includegraphics[scale=0.2]{filter/noir.jpg}
\caption{No filter image}
\end{subfigure}
\begin{subfigure}{0.5\textwidth}
\centering
\includegraphics[scale=0.17]{filter/blue_ir.jpg}
\caption{Blue filter image}
\end{subfigure}
\begin{subfigure}{0.5\textwidth}
\centering
\includegraphics[scale=0.17]{filter/green_ir.jpg}
\caption{Green filter image}
\end{subfigure}
\begin{subfigure}{0.5\textwidth}
\centering
\includegraphics[scale=0.17]{filter/red_ir.jpg}
\caption{Red filter image}
\end{subfigure}
\caption{Filter images}
\label{fig:filters}
\end{figure}

In Figure \ref{fig:filters_ndvis}, the $=$ sign denotes which channel the $NIR$ and $VIS$ of Equation \ref{eq:ndvi} uses.

\begin{figure}[H]
\begin{subfigure}{0.5\textwidth}
\centering
\includegraphics[scale=0.17]{filter/noir_Color_Index.jpg}
\caption{No filter NDVI (NIR=red, VIS=blue)}
\end{subfigure}
\begin{subfigure}{0.5\textwidth}
\centering
\includegraphics[scale=0.17]{filter/noir_d_ndvi.jpg}
\caption{No filter NDVI (NIR=red, VIS=red)}
\end{subfigure}
\begin{subfigure}{0.5\textwidth}
\centering
\includegraphics[scale=0.17]{filter/blue_Color_Index.jpg}
\caption{Blue filter NDVI (NIR=red, VIS=blue)}
\end{subfigure}
\begin{subfigure}{0.5\textwidth}
\centering
\includegraphics[scale=0.17]{filter/blue_d_ndvi.jpg}
\caption{Blue filter NDVI (NIR=red, VIS=red)}
\end{subfigure}
\begin{subfigure}{0.5\textwidth}
\centering
\includegraphics[scale=0.17]{filter/green_Color_Index.jpg}
\caption{Green filter NDVI (NIR=red, VIS=green)}
\end{subfigure}
\begin{subfigure}{0.5\textwidth}
\centering
\includegraphics[scale=0.17]{filter/green_d_ndvi.jpg}
\caption{Green filter NDVI (NIR=red, VIS=red)}
\end{subfigure}
\begin{subfigure}{0.5\textwidth}
\centering
\includegraphics[scale=0.17]{filter/red_Color_Index.jpg}
\caption{Red filter NDVI (NIR=blue, VIS=red)}
\end{subfigure}
\begin{subfigure}{0.5\textwidth}
\centering
\includegraphics[scale=0.17]{filter/red_d_ndvi.jpg}
\caption{Red filter NDVI (NIR=blue, VIS=red)}
\end{subfigure}
\caption{NDVIs of single (left column) and stereo (right column) filter images}
\label{fig:filters_ndvis}
\end{figure}

\chapter{NDVI Map}
\label{app:ndvi_map}

A large map was produced. The young leaves could be discerned, and had values between 0.5-0.55. This is great news, because the Parrot Sequoia could not discern the leaves having an infra-red camera resolution of only 1.2 MP, as opposed to 8 MP in our application.\\

Cloud processing was used. Final floating-point tiff image is 8842x8283 pixels, at 286 Mb. The processing used 24364 Mb of RAM.\\

\begin{figure}[H]
\begin{subfigure}{0.5\textwidth}
\centering
\includegraphics[scale=0.6]{images/map_rgb_large.png}
\caption{RGB map}
\end{subfigure}
\begin{subfigure}{0.5\textwidth}
\centering
\includegraphics[scale=0.6]{images/map_ir_large.png}
\caption{NIR map}
\end{subfigure}
\caption{Maps processed using MME \cite{mme}}
\label{fig:app_maps}
\end{figure}

\begin{figure}[H]
\centering
\includegraphics[scale=0.35]{images/map_ndvi_large.png}
\caption{NDVI performed using bUnwarpJ \cite{bunwarpj}}
\label{fig:app_stitch_map}
\end{figure}

\chapter{Chessboard input set}
\label{app:chess}

\begin{figure}[H]
\centering
\includegraphics[scale=0.505]{images/rgb_chess_set.JPG}
\caption{RGB input chess images}
\label{fig:rgb_input_chess}
\end{figure}

\begin{figure}[H]
\centering
\includegraphics[scale=0.5]{images/ir_chess_set.JPG}
\caption{NIR input chess images}
\label{fig:ir_input_chess}
\end{figure}

\chapter{Stereo calibration and rectification results}
\label{app:stereo_results}

\begin{equation}\label{eq::cm1}
M_c^1 = \begin{bmatrix}
2.523278 & 0 & 1.64 \\
0 & 2.516529 & 1.232 \\
0 & 0 & 0.001
\end{bmatrix}\cdot10^3
\end{equation}

\begin{equation}\label{eq::cm2}
M_c^2 = \begin{bmatrix}
2.505381 & 0 & 1.64 \\
0 & 2.505083 & 1.232 \\
0 & 0 & 0.001
\end{bmatrix}\cdot10^3
\end{equation}

\begin{equation}\label{eq::R}
R = \begin{bmatrix}
999.281 & -6.4992 & 37.35413 \\
6.59547 & 999.97524 & -2.45456 \\
-37.33725 & 2.69917 & 999.29908
\end{bmatrix}\cdot10^3
\end{equation}

\begin{equation}\label{eq::E}
E = \begin{bmatrix}
-0.3150989 & 158.311619 & 35.973848 \\
-55.293721 & -6.403995 & -2757.525727 \\
-18.200605 & 2753.713987 & -8.117986
\end{bmatrix}\cdot10^-3
\end{equation}

\begin{equation}\label{eq::F}
F = \begin{bmatrix}
-2.54558709\cdot10^-6 & 1.28238104\cdot10^-3 & -0.842399343\\
-0.446754\cdot10^-3 & -51.880818\cdot10^-6 & -55.4216639\\
0.1861918 & 53.846 & 1000
\end{bmatrix}\cdot10^-3
\end{equation}

\begin{equation}\label{eq::R1}
R_1 = \begin{bmatrix}
0.99977728 & 0.00654886 & -0.02006268\\
0.00657528 & 0.9999776 & -0.00125126\\
0.02005403 & 0.0013829 & 0.99979794
\end{bmatrix}
\end{equation}

\begin{equation}\label{eq::R2}
R_2 = \begin{bmatrix}
0.99826641 & 0.01319195 & -0.05735987\\
-0.01311635 & 0.99991254 & 0.00169423\\
0.05737721 & -0.00093894 & 0.99835213
\end{bmatrix}
\end{equation}

\begin{equation}\label{eq::P1}
P_1 = \begin{bmatrix}
2505.083324 & 0 & 1778.704819 & 0\\
0 & 2505.083324 & 1231.092529  & 0\\
0 & 0 & 1 & 0
\end{bmatrix}
\end{equation}

\begin{equation}\label{eq::P2}
P_2 = \begin{bmatrix}
2505.083324 &    0.       & 1778.704819 & 6909.840227\\
 0          & 2505.083324 & 1231.092529 &    0        \\
 0          &    0        &    1        &    0         \\
\end{bmatrix}
\end{equation}

\begin{equation}\label{eq::Q}
Q\ =\ \begin{bmatrix}
1 & 0 & 0 & -1.778705\cdot10^3\\
0 & 1 & 0 & -1.231093\cdot10^3\\
0 & 0 & 0 & 2.505083\cdot10^3\\
0 & 0 & -0.362539 & 0
\end{bmatrix}
\end{equation}

\begin{equation}\label{eq::T}
T\ =\ \begin{bmatrix}
2.75354568\\
0.03638771\\
-0.15821732
\end{bmatrix}
\end{equation}

\begin{equation}\label{eq::validpix1}
valid\ Pix\ ROI_1\ =\ 
\begin{bmatrix}
x_1 & y_1\\
x_2 & y_2
\end{bmatrix}\ =
\begin{bmatrix}
88 & 6\\
3192 & 2421
\end{bmatrix}
\end{equation}

\begin{equation}\label{eq::validpix2}
valid\ Pix\ ROI_2\ =\ 
\begin{bmatrix}
x_1 & y_1\\
x_2 & y_2
\end{bmatrix}\ =
\begin{bmatrix}
0 & 26\\
3204 & 2378
\end{bmatrix}
\end{equation}

\begin{equation}\label{eq::error1}
error_1 = 17.255\%
\end{equation}
\begin{equation}\label{eq::error2}
error_2 = 14.647\%
\end{equation}

\chapter{Project Planning}

% Please add the following required packages to your document preamble:
% \usepackage{graphicx}
\begin{table}[H]
\centering
%\caption{My caption}
\label{my-label}
\resizebox{\textwidth}{!}{%
\begin{tabular}{ll}
\textbf{Task Description}             & \textbf{Duration}       \\
Drone construction                    & 7-15 July 2017          \\
First flight attempts                 & 15-17 July 2017         \\
Official commencement of Project E448 & 17 July 2017            \\
Drone calibration, software           & 18 July - 7 August 2017 \\
Camera acquisition                    & 8 August - 3 Sept 2017  \\
Camera calibration                    & 4 Sept - 16 Oct 2017    \\
Flight with Chris on Longridge        & 29 Sept 2017            \\
Image processing                      & 17 - 29 Oct 2017        \\
Report writing                        & 11 Oct 2017             \\
Hand in report                        & 30 Oct 2017            
\end{tabular}%
}
\end{table}


\chapter{Outcome Compliance}

% Please add the following required packages to your document preamble:
% \usepackage{graphicx}
\begin{table}[H]
\centering
%\caption{My caption}
\label{my-label}
\resizebox{\textwidth}{!}{%
\begin{tabular}{ll}
\textbf{Outcome} & \textbf{Criteria} \\
ELO 1. Problem Solving & Chapter 3, 4, 5, 6, Appendices \\
ELO 2. Application of scientific and engineering knowledge & Chapter 2, 3, 4, 5, 6, Appendices \\
ELO 3. Engineering Design & Chapter 3, 4, 5, Appendices \\
ELO 4. Investigations, experiments and data analysis & Chapter 2, 3, 4, 5, 6, Appendices \\
\begin{tabular}[c]{@{}l@{}}ELO 5. Engineering methods, skills and tools, \\ including Information Technology\end{tabular} & Chapter 3, 4, 5, 6, Appendices \\
ELO 6. Professional and technical communication & Chapter 1, 2, 3, 4, 5, 6 \\
ELO 9. Independent Learning Ability & Chapter 2, 3, 4, 5, 6, Appendices
\end{tabular}%
}
\end{table}


\chapter{Code}

Code block \ref{code:getmatrices} shows the method to get camera matrices.

\lstset{language=python,caption={Get matrices},label=code:getmatrices}
\begin{lstlisting}
import cv2 #, numpy, matplotlib
import numpy as np
from matplotlib import pyplot as plt
import glob
import dill
from tqdm import tqdm
import os

##img_type = 'ir'
img_type = 'rgb'
##img_type = 'example'

_x = 15
_y = 15

savefile = img_type + '_matrices_15_dual.pkl'
##root_path = 'C:/Users/d7rob/thesis/chess/master_set/'
root_path = 'C:/Users/d7rob/thesis/chess/15_dual_ethernet/rgb/'
##root_path = 'chess/'
##images = glob.glob('chess/*.jpg')
##images = glob.glob('L:/Backups/thesis/chess/rgb/compressed/picked/*.jpg')
##images = glob.glob('C:/Users/d7rob/thesis/chess/master_set/rgb_7x6/*.jpg')
##images = glob.glob(root_path + img_type + '_7x6/*.jpg')
##images = glob.glob(root_path + img_type + '_15x15/*.jpg')
images = glob.glob(root_path + '/*.jpg')

##gray = 0
# termination criteria
criteria = (cv2.TERM_CRITERIA_EPS + cv2.TERM_CRITERIA_MAX_ITER, 30, 0.001)
# prepare object points, like (0,0,0), (1,0,0), (2,0,0) ....,(6,5,0)
objp = np.zeros((_y*_x,3), np.float32)
objp[:,:2] = np.mgrid[0:_x,0:_y].T.reshape(-1,2)
# Arrays to store object points and image points from all the images.
objpoints = [] # 3d point in real world space
imgpoints = [] # 2d points in image plane.

for fname in tqdm(images):
    print fname
    img = cv2.imread(fname)
    gray = cv2.cvtColor(img, cv2.COLOR_BGR2GRAY)
    # Find the chess board corners
    
    flags = cv2.CALIB_CB_ADAPTIVE_THRESH + \
            cv2.CALIB_CB_NORMALIZE_IMAGE + \
            cv2.CALIB_CB_FILTER_QUADS + \
            cv2.CALIB_CB_FAST_CHECK
##    ret, corners = cv2.findChessboardCorners(gray, (_x,_y), flags)

    ret, corners = cv2.findChessboardCorners(gray, (_x,_y), None, flags)
    # If found, add object points, image points (after refining them)
    if ret == True:
        objpoints.append(objp)
        corners2=cv2.cornerSubPix(gray,corners, (11,11), (-1,-1), criteria)
        imgpoints.append(corners2)
        # Draw and display the corners
        cv2.drawChessboardCorners(img, (_x,_y), corners2, ret)
        draw_path = root_path + os.path.basename(fname)[:-4] + '_'+str(_x)+'x'+str(_y)+'_drawn.jpg'
        print draw_path
        cv2.imwrite(draw_path, img)
##        cv2.imshow('img', img)
##        cv2.waitKey(500)
cv2.destroyAllWindows()

ret, mtx, dist, rvecs, tvecs = cv2.calibrateCamera(objpoints, imgpoints, gray.shape[::-1], None, None)
print ret, mtx, dist, rvecs, tvecs
##img = cv2.imread('C:/Users/d7rob/thesis/chess/rgb/img1_2017-09-25_23-18-52.001_1.jpg')
####img = cv2.imread('chess/left12.jpg')
##h,  w = img.shape[:2]
##newcameramtx, roi=cv2.getOptimalNewCameraMatrix(mtx, dist, (w,h), 1, (w,h))

mean_error = 0
for i in xrange(len(objpoints)):
    imgpoints2, _ = cv2.projectPoints(objpoints[i], rvecs[i], tvecs[i], mtx, dist)
    error = cv2.norm(imgpoints[i], imgpoints2, cv2.NORM_L2)/len(imgpoints2)
    mean_error += error
total_error = mean_error/len(objpoints)
print( "total error: {}".format(total_error) )

dill.dump_session(savefile)

\end{lstlisting}

\newpage
Code block \ref{code:sortcorners} shows the method to sort corner pics from unsuccessful ones.

\lstset{language=python,caption={Sorting potential corner images},label=code:sortcorners}
\begin{lstlisting}
import glob, cv2
import numpy as np
from matplotlib import pyplot as plt

##images = glob.glob('L:/Backups/thesis/chess/rgb/compressed2/picked/*.jpg')
root_dir = 'C:/Users/d7rob/thesis/chess/15_dual_ethernet/compressed'
images = glob.glob(root_dir + '/*.jpg')
print images

for fname in images:
    img = cv2.imread(fname)

    ##gray = cv2.cvtColor(img, cv2.COLOR_RGB2GRAY)
    gray = cv2.cvtColor(img, cv2.COLOR_BGR2GRAY)
    ##gray = img

    ##plt.imshow(gray)
    ##plt.show()

    ##cv2.imshow('gray', gray)
    ##cv2.waitKey(500)
    ##cv2.destroyAllWindows()

    flags = cv2.CALIB_CB_ADAPTIVE_THRESH + \
            cv2.CALIB_CB_NORMALIZE_IMAGE + \
            cv2.CALIB_CB_FILTER_QUADS + \
            cv2.CALIB_CB_FAST_CHECK
##    flags = cv2.CALIB_CB_FAST_CHECK
    ##flags = None
    # 7, 6
##    ret, corners = cv2.findChessboardCorners(gray, (15,15), flags)
    ret, corners = cv2.findChessboardCorners(gray, (15,15), None, flags)

    print np.shape(corners), ret

    if ret == True:
        pass
##        cv2.imwrite((str(fname[:-4]) + '_has_15x15_corners.jpg'), img)

\end{lstlisting}

\newpage
Code block \ref{code:stereorectification} shows the method to stereorectify.

\lstset{language=python,caption={Stereorectification},label=code:stereorectification}
\begin{lstlisting}
import numpy as np
import dill, glob, tqdm

np.set_printoptions(suppress=True)

root_dir = 'C:/Users/d7rob/thesis/distorted'
root_dir = 'L:/Backups/thesis/longridge'

_rgb_name = "/rgb"
_ir_name = "/ir"
rgb_images = glob.glob(root_dir + _rgb_name + '/*.jpg')
ir_images = glob.glob(root_dir + _ir_name + '/*.jpg')
zipped_images = zip(rgb_images, ir_images)

##print ": " + str()
def pp(val):
    print '::\t' + str(val)

class Matrices:
    def __init__(self, total_error, ret, mtx, dist, rvecs, tvecs, imgpoints, objpoints, gray):
        self.total_error = total_error
        self.ret = ret
        self.mtx = mtx
        self.dist = dist
        self.rvecs = rvecs
        self.tvecs = tvecs
        self.imgpoints = imgpoints
        self.objpoints = objpoints
        self.resolution = gray.shape[::-1]

    def print_values(self):
        print 'total_error:\t\t' + str(self.total_error)
        print 'ret:\t\t\t' + str(self.ret)
        print 'mtx.shape:\t\t' + str(self.mtx.shape)
        print 'dist.shape:\t\t' + str(self.dist.shape)
        print 'rvecs.shape:\t\t' + str(np.shape(self.rvecs))
        print 'tvecs.shape:\t\t' + str(np.shape(self.tvecs))
        print 'np.shape(objpoints):\t' + str(np.shape(self.objpoints))
        print 'np.shape(imgpoints):\t' + str(np.shape(self.imgpoints))
        print 'resolution:\t\t' + str(self.resolution)
        print ''
##        print 'imgpoints2.shape:\t' + str(np.shape(imgpoints2))

##savefile = 'rgb_matrices.pkl'
##savefile = 'rgb_matrices_13_print.pkl'
##savefile = 'rgb_matrices_14_dual.pkl'
##savefile = '_matrices_12_ESS.pkl'
savefile = 'rgb_matrices_15_dual.pkl'
dill.load_session(savefile)
rgb = Matrices(total_error, ret, mtx, dist, rvecs, tvecs, imgpoints, objpoints, gray)
rgb.print_values()

##savefile = 'ir_matrices.pkl'
##savefile = 'ir_matrices_13_print.pkl'
##savefile = '_matrices_12_ESS.pkl'
savefile = 'ir_matrices_15_dual.pkl'
dill.load_session(savefile)
ir = Matrices(total_error, ret, mtx, dist, rvecs, tvecs, imgpoints, objpoints, gray)
ir.print_values()

##### find fundamental matrix F
####F, mask = cv2.findFundamentalMat(np.array(rgb.imgpoints[0]), np.array(ir.imgpoints[0]))
####print "F: " + str(F)
####print "mask.shape: " + str(mask.shape)
####print ''
####
##### uncalibrated stereo rectification
####ret, H1, H2 = cv2.stereoRectifyUncalibrated(rgb.imgpoints[0], ir.imgpoints[0], F, rgb.resolution)
####print "stereoRectifyUncalibrated ret: " + str(ret)
####print "H1: " + str(H1)
####print "H2: " + str(H2)
####
####R1 = np.linalg.inv(rgb.mtx)*H1*rgb.mtx
####R2 = np.linalg.inv(ir.mtx)*H2*ir.mtx
####P1 = rgb.mtx
####P2 = ir.mtx
####
####print "R1: " + str(R1)
####print "R2: " + str(R2)
####print "P1: " + str(P1)
####print "P2: " + str(P2)
####print ''

# mono calibration

ret, mtx, dist, rvecs, tvecs = cv2.calibrateCamera(rgb.objpoints, rgb.imgpoints, rgb.resolution, None, None)

# stereo calibration
flags = cv2.CALIB_FIX_ASPECT_RATIO + \
                    cv2.CALIB_ZERO_TANGENT_DIST + \
                    cv2.CALIB_USE_INTRINSIC_GUESS + \
                    cv2.CALIB_SAME_FOCAL_LENGTH + \
                    cv2.CALIB_RATIONAL_MODEL + \
                    cv2.CALIB_FIX_K3 + cv2.CALIB_FIX_K4 + cv2.CALIB_FIX_K5

criteria = (cv2.TERM_CRITERIA_COUNT + cv2.TERM_CRITERIA_EPS, 100, 0.00001)

initCameraMatrix1 = cv2.initCameraMatrix2D(rgb.objpoints, rgb.imgpoints, rgb.resolution, 0);
initCameraMatrix2 = cv2.initCameraMatrix2D(ir.objpoints, ir.imgpoints, ir.resolution, 0);
initDist = np.array([[0]*5])
print "initCameraMatrix1: " + str(initCameraMatrix1)
print "initCameraMatrix2: " + str(initCameraMatrix2)
    
##retval, cameraMatrix1, distCoeffs1, cameraMatrix2, distCoeffs2, R, T, E, F = cv2.stereoCalibrate(rgb.objpoints, rgb.imgpoints, ir.imgpoints, initCameraMatrix1, initDist, initCameraMatrix2, initDist, rgb.resolution, flags, criteria)

retval, cameraMatrix1, distCoeffs1, cameraMatrix2, distCoeffs2, R, T, E, F = cv2.stereoCalibrate(rgb.objpoints, \
                                                                                                 rgb.imgpoints, \
                                                                                                 ir.imgpoints, \
                                                                                                 rgb.resolution,
                                                                                                 initCameraMatrix1, \
                                                                                                 initDist, \
                                                                                                 initCameraMatrix2, \
                                                                                                 initDist, \
                                                                                                 None, None, None, None, \
                                                                                                 criteria, \
                                                                                                 flags)

print "stereoCalibrate retval: " + str(retval)
print "cameraMatrix1: " + str(cameraMatrix1)
print "distCoeffs1: " + str(distCoeffs1)
print "cameraMatrix2: " + str(cameraMatrix2)
print "distCoeffs2: " + str(distCoeffs2)
print "R: " + str(R)
print "T: " + str(T)
print "E: " + str(E)
print "F: " + str(F)
print ''

# stereo rectification
flags = cv2.CALIB_ZERO_DISPARITY
##flags = None

##R1, R2, P1, P2, Q, validPixROI1, validPixROI2 = cv2.stereoRectify(cameraMatrix1, distCoeffs1, cameraMatrix2, distCoeffs2, rgb.resolution, R, T, flags, 1, rgb.resolution)

R1, R2, P1, P2, Q, validPixROI1, validPixROI2 = cv2.stereoRectify(cameraMatrix1, distCoeffs1, cameraMatrix2, distCoeffs2, rgb.resolution, R, T, None, None, None, None, None, flags, 1, rgb.resolution)

print "R1: " + str(R1)
print "R2: " + str(R2)
print "P1: " + str(P1)
print "P2: " + str(P2)
print "Q: " + str(Q)
print "validPixROI1: " + str(validPixROI1)
print "validPixROI2: " + str(validPixROI2)
print ''

# pre-compute undistortion matrices
rgb_mapx, rgb_mapy = cv2.initUndistortRectifyMap(cameraMatrix1, distCoeffs1, R1, P1, rgb.resolution, cv2.CV_32FC1) # 5
ir_mapx, ir_mapy = cv2.initUndistortRectifyMap(cameraMatrix2, distCoeffs2, R2, P2, ir.resolution, cv2.CV_32FC1) # 5

# batch undistortion

for rgb_path, ir_path in tqdm(zipped_images):
    print ''
    rgb_img = cv2.imread(rgb_path)
    dst = cv2.remap(rgb_img, rgb_mapx, rgb_mapy, cv2.INTER_LINEAR)
    rgb_undistort_path = rgb_path[:-4] + '_stereo_undistorted.jpg'
##    print "rgb_undistort_path: " + str(rgb_undistort_path)
##    cv2.rectangle(dst, validPixROI1[:2], validPixROI1[2:],(0, 0, 255),30)
    dst = dst[validPixROI1[1]:validPixROI1[3], validPixROI1[0]:validPixROI1[2]]
    ##$ cv2.imwrite(rgb_undistort_path, dst)

    ir_img = cv2.imread(ir_path)
    dst = cv2.remap(ir_img, ir_mapx, ir_mapy, cv2.INTER_LINEAR)
    ir_undistort_path = ir_path[:-4] + '_stereo_undistorted.jpg'
##    print "ir_undistort_path: " + str(ir_undistort_path)
##    cv2.rectangle(dst, validPixROI2[:2], validPixROI2[2:],(0, 0, 255),30)
    dst = dst[validPixROI2[1]:validPixROI2[3], validPixROI2[0]:validPixROI2[2]]
    ##$ cv2.imwrite(ir_undistort_path, dst)

    # mono output

####    h, w = rgb_img.shape[:2]
####    newcameramtx, roi = cv2.getOptimalNewCameraMatrix(mtx, dist, (w,h), 1, (w,h))
####    print "newcameramtx: " + str(newcameramtx)
####    print "roi: " + str(roi)
####
####    # undistort mono
####    _mapx, _mapy = cv2.initUndistortRectifyMap(mtx, dist, None, newcameramtx, (w,h), cv2.CV_32FC1) # 5
####    _path = rgb_path
####    _img = cv2.imread(_path)
####    _dst = cv2.remap(_img, _mapx, _mapy, cv2.INTER_LINEAR)
####
####    _undistort_path = _path[:-4] + '_mono_undistorted.jpg'
####    cv2.imwrite(_undistort_path, _dst)

\end{lstlisting}

\newpage
Code block \ref{code:duallistmaker} shows the method to create a dual list.

\lstset{language=python,caption={Dual list creator},label=code:duallistmaker}
\begin{lstlisting}
import glob

root_dir = 'C:/Users/d7rob/thesis/distorted/undistorted'
root_dir = 'L:/Backups/thesis/longridge'
root_dir = 'C:/Users/d7rob/thesis/home_lab_window_2'

_rgb_name = "/rgb_blue"
_ir_name = "/ir_blue"
rgb_images = glob.glob(root_dir + _rgb_name + '/*.jpg')
ir_images = glob.glob(root_dir + _ir_name + '/*.jpg')

for x, y in zip(rgb_images, ir_images):
    print y + ', ' + x

\end{lstlisting}


Code block \ref{code:tcp_trigger} shows the method to trigger simultaneously via TCP.

\lstset{language=python,caption={TCP triggering},label=code:tcp_trigger}
\begin{lstlisting}
import socket, os, threading
from threading import Thread
from time import sleep, time

clients = set()
clients_lock = threading.Lock()
camera_mode = False
cameras_triggered = 0
time_now = 0


def listener(_client, _address):
    global camera_mode, cameras_triggered
    print("Accepted connection from: ", _address)
    cameras_triggered += 1
    with clients_lock:
        clients.add(_client)
    try:
        while True:
            data = _client.recv(32)
            print("received: " + repr(data))
            if not data:
                break
            if "CAMERA_MODE_ENABLED" in data:
                camera_mode = True
                for c in clients:
                    c.sendall("ACK")
            if "CAMERA_MODE_DISABLED" in data:
                camera_mode = False
                for c in clients:
                    c.sendall("ACK")
            if "CAMERA_COMPLETE":
                cameras_triggered += 1
    except Exception as e:
        print(e)
    finally:
        with clients_lock:
            clients.remove(_client)
            _client.close()

count = 0


def camera_process():
    global camera_mode, cameras_triggered, clients, time_now, count
    print("Trigger thread ready.")
    while True:
        if camera_mode and cameras_triggered >= len(clients) and len(clients) > 0:
            cameras_triggered = 0
            with clients_lock:
                print("Sending trigger " + repr(count) + " to " + repr(len(clients)) + " clients; last " + repr(time() - time_now) + ".")
                count += 1
                time_now = time()
                for c in clients:
                    c.sendall("TRIGGER_NOW")
        sleep(0.001)

t = Thread(target=camera_process)
t.daemon = True
t.start()

#host = 'localhost'
host = ''
port = 10017

s = socket.socket()
s.setsockopt(socket.SOL_SOCKET, socket.SO_REUSEADDR, 1)
s.bind((host, port))
s.listen(3)
th = []

try:
    while True:
        print("Server is listening for connections...")
        client, address = s.accept()
        print ("hi")
        t = Thread(target=listener, args=(client, address))
        t.daemon = True
        th.append(t.start())
except KeyboardInterrupt as e:
    print(e)
\end{lstlisting}

Code block \ref{code:init_mav} shows the method to initialize mavlink.

\lstset{language=bash,caption={Initialize mavlink},label=code:init_mav}
\begin{lstlisting}
#!/bin/bash

echo "MAVlink initialisation started."

arg1=
start=true
PREV_IP_GCS=
while true ; do
	sleep 0.5
	IP_GCS="`ping -I wlan0 -c1 rod693a | sed -nE 's/^PING[^(]+\(([^)]+)\).*/\1/p'`"
	if ((${#IP_GCS})); then
		printf "Ground Control Station IP address is %s\n" "$IP_GCS"
		#arg1="--out $IP_GCS:14550"
		arg1="-e $IP_GCS:14550"
		if ! [ "$IP_GCS" = "$PREV_IP_GCS" ] ; then start=true ; fi
		PREV_IP_GCS=$IP_GCS
	else
		unset arg1
	fi
	if [ "$start" = true ] ; then
		#sudo pkill -f "mavproxy.py"
		sudo pkill -f "mavlink-routerd"
		#cmd="mavproxy.py --daemon --master localhost:14550 --out localhost:14850 --out 169.254.169.254:14850 $arg1"
		cmd="/home/pi/mavlink-router/mavlink-routerd 0.0.0.0:14550 -e localhost:14850 -e 169.254.169.254:14850 $arg1"
		echo $cmd
		$cmd &
		start=false
	fi
	sleep 4.5
done

\end{lstlisting}

Code block \ref{code:wifiup} shows the method to reconnect wifi.

\lstset{language=bash,caption={Wifi reconnnection},label=code:wifiup}
\begin{lstlisting}
#!/bin/bash

echo "Wifi reconnection monitor started."

while true ; do
  if iwconfig wlan0 | grep -q "ESSID:off" ; then
    echo "Network connection down! Attempting reconnection."
    ifup --force wlan0
  fi
  sleep 5
done

\end{lstlisting}

Code block \ref{code:rc_local} shows the method to boot scripts.

\lstset{language=bash,caption={Scripts on-boot},label=code:rc_local}
\begin{lstlisting}
#!/bin/sh -e
#
# rc.local
#
# This script is executed at the end of each multiuser runlevel.
# Make sure that the script will "exit 0" on success or any other
# value on error.
#
# In order to enable or disable this script just change the execution
# bits.
#
# By default this script does nothing.

# Print the IP address
_IP=$(hostname -I) || true
if [ "$_IP" ]; then
  printf "My IP address is %s\n" "$_IP"
fi

su - pi -c "screen -dmS wifi sudo /home/pi/re_wlan.sh"
su - pi -c "screen -dmS mav ~/init_mav.sh"
su - pi -c "screen -dmS cam python ~/thesis-image-processing/dronekit/capture_mode.py"

exit 0

\end{lstlisting}

Code block \ref{code:interfaces} shows the method interfaces file.

\lstset{language=bash,caption={Interfaces file},label=code:interfaces}
\begin{lstlisting}
# interfaces(5) file used by ifup(8) and ifdown(8)

# Please note that this file is written to be used with dhcpcd
# For static IP, consult /etc/dhcpcd.conf and 'man dhcpcd.conf'

# Include files from /etc/network/interfaces.d:
source-directory /etc/network/interfaces.d

auto lo
iface lo inet loopback

allow-hotplug intwifi0
iface intwifi0 inet manual
    wpa-conf /etc/wpa_supplicant/wpa_supplicant.conf

allow-hotplug wlan0
iface wlan0 inet manual
    wpa-conf /etc/wpa_supplicant/wpa_supplicant.conf

allow-hotplug wlan1
iface wlan1 inet manual
    wpa-roam /etc/wpa_supplicant/wpa_supplicant.conf

allow-hotplug eth0
iface eth0 inet manual

iface default inet dhcp

\end{lstlisting}

Code block \ref{code:arducopter} shows the method to connect arducopter.

\lstset{language=bash,caption={Arducopter connections file},label=code:arducopter}
\begin{lstlisting}
# Default settings for ArduPilot for Linux.
# The file is sourced by systemd from arducopter.service

#TELEM1="-A tcp:127.0.0.1:5763:wait"
TELEM1="-A udp:127.0.0.1:14550"
TELEM2="-C /dev/ttyAMA0"
TELEM3="-D udp:10.0.0.95:14550"
#TELEM3="-D udp:0.0.0.0:14250"
TELEM4="-C /dev/ttyUSB0"
TELEM5="-C /dev/ttyUSB1"

# Options to pass to ArduPilot
ARDUPILOT_OPTS="$TELEM1 $TELEM2 $TELEM3" #$TELEM4 $TELEM5"

                          #    # ###### #      #####
                          #    # #      #      #    #
                          ###### #####  #      #    #
                          #    # #      #      #####
                          #    # #      #      #
                          #    # ###### ###### #

# -A is a console switch (usually this is a Wi-Fi link)

# -C is a telemetry switch
# Usually this is either /dev/ttyAMA0 - UART connector on your Navio
# or /dev/ttyUSB0 if you're using a serial to USB convertor

# -B or -E is used to specify non default GPS

# Type "emlidtool ardupilot" for further help
\end{lstlisting}

Code block \ref{code:simultaneous} shows the method to trigger simultaneous shots.

\lstset{language=python,caption={Potential packet},label=code:simultaneous}
\begin{lstlisting}
from dronekit import connect, VehicleMode
from threading import Thread
from time import sleep, time
# import datetime as dt
from datetime import datetime, timedelta
import socket, os, threading, sys
import picamera

UDP_IP = "169.254.169.254"
UDP_PORT = 5005

# Connect to UDP endpoint.
connecting_bool = True
while connecting_bool:
  try:
    print "Connecting to UDP endpoint."
    vehicle = connect('0.0.0.0:14850', wait_ready=True)
    if vehicle: connecting_bool = False
  except KeyboardInterrupt:
    connecting_bool = False
    sys.exit(0)
  except Exception as e:
    print e

camera = picamera.PiCamera()
camera.resolution = (3280, 2464)
camera.start_preview()

print " Mode: %s" % vehicle.mode.name
print vehicle.channels['1']

##print "Autopilot Firmware version: %s" % vehicle.version
##print "Autopilot capabilities (supports ftp): %s" % vehicle.capabilities.ftp
##print "Global Location: %s" % vehicle.location.global_frame
##print "Global Location (relative altitude): %s" % vehicle.location.global_relative_frame
##print "Local Location: %s" % vehicle.location.local_frame    #NED
##print "Attitude: %s" % vehicle.attitude
##print "Velocity: %s" % vehicle.velocity
##print "GPS: %s" % vehicle.gps_0
##print "Groundspeed: %s" % vehicle.groundspeed
##print "Airspeed: %s" % vehicle.airspeed
##print "Gimbal status: %s" % vehicle.gimbal
##print "Battery: %s" % vehicle.battery
##print "EKF OK?: %s" % vehicle.ekf_ok
##print "Last Heartbeat: %s" % vehicle.last_heartbeat
##print "Rangefinder: %s" % vehicle.rangefinder
##print "Rangefinder distance: %s" % vehicle.rangefinder.distance
##print "Rangefinder voltage: %s" % vehicle.rangefinder.voltage
##print "Heading: %s" % vehicle.heading
##print "Is Armable?: %s" % vehicle.is_armable
##print "System status: %s" % vehicle.system_status.state
##print "Mode: %s" % vehicle.mode.name    # settable
##print "Armed: %s" % vehicle.armed    # settable

#@vehicle.on_attribute('location')
#def listener(self, attr_name, value):
    # `self`: :py:class:`Vehicle` object that has been updated.
    # `attr_name`: name of the observed attribute - 'location'
    # `value` is the updated attribute value (a :py:class:`Locations`). This can be queried for the frame information
    #print " Global: %s" % value.global_frame
    #print " GlobalRelative: %s" % value.global_relative_frame
    #print " Local: %s" % value.local_frame

@vehicle.on_message('MAV_CMD_DO_DIGICAM_CONTROL')
def my_method(self, name, msg):
    print " name: %s" % name
    print " msg: %s" % msg

@vehicle.on_message('CAMERA_TRIGGER')
def my_method2(self, name, msg):
    print " name: %s" % name
    print " msg: %s" % msg

last_trigger_time = 0
cam_count = 0
trigger_count = 0
@vehicle.on_message('CAMERA_FEEDBACK')
def process_camera(self, name, msg):
    global last_trigger_time, camera, trigger_count
    trigger_count += 1
    if (trigger_count % 2):
        pass
      #print msg.time_usec
      ##take_photo(msg.time_usec)
    #if time() - last_trigger > 1.0/2:
        #print " name: %s" % name
        #print " msg: %s" % msg
        #print msg.time_usec
    #   last_trigger = time()
    #   missed=0
    #    cam_count += 1
    #else:
    #   if (missed % 2 > 0):
    #       print "missed_" + datetime.datetime.now().strftime('%Y-%m-%d_%H-%M-%S.%f')[:-3]
    #   missed += 1
    
def wait():
    # Calculate the delay to the start of the next hour
    next_sec = (datetime.now() + timedelta(seconds=1)).replace(microsecond=0)
    delay = (next_sec - datetime.now()).microseconds/1000000.0
    # print "delay: " + repr(delay)
    sleep(delay)

def take_photo():
    global cam_count
    cam_count += 1
    dir = '/home/pi/images/'
    os.system("mkdir -p " + dir)
    wait()
    string = dir + 'img_' + datetime.now().strftime('%Y-%m-%d_%H-%M-%S.%f')[:-3] + '_rgb_' + str(cam_count) + '.jpg'
    print string
    #string2 = dir + 'img2_' + time.strftime("%Y-%m-%d_%H-%M-%S_", time.gmtime(t/1000000)) + str(t)[-6:-3] + '_' + str(cam_count) + '.jpg'
    # print string2
    # sleep(0.5)
    camera.capture(string)
    # time.sleep(0.2)

"""
sock = socket.socket()
ack = False
connected = False


def process_data():
    global sock, ack, connected
    while not connected:
        try:
            sock = socket.socket()
            if not sock.connect(('localhost', 10017)):
              connected = True
        except Exception as e:
          print(e)
    try:
        while True:
            data = sock.recv(32)
            # print("data: " + repr(data))
            if "KEEP_ALIVE" in data:
                sock.sendall("KEEP_ALIVE")
            if "ACK" in data:
                ack = True
            if "TRIGGER_NOW" in data and camera_mode:
                take_photo(0)
                sock.sendall("CAMERA_COMPLETE1")
    except Exception as e:
        print(e)
    finally:
        sock.close()
"""

#Create a message listener for all messages.
#@vehicle.on_message('*')
#def listener(self, name, message):
#    print name
#    if 'CAMERA_FEEDBACK' in name:
# print message
#    print 'message: %s' % message

"""
t = Thread(target=process_data)
t.daemon = True
t.start()
"""

camera_mode = False
prev_cam_mode = False
# cam_count = 0

try:
  while True:
      camera_mode = vehicle.channels['8'] > 1100# or True
      if camera_mode:
        take_photo()
      else:
        cam_count = 0
      # if camera_mode != prev_cam_mode:
      #   ack = False
      # if camera_mode:
      #   if not ack and connected:
      #     sock.sendall("CAMERA_MODE_ENABLED1")
      # else:
      #   if not ack and connected:
      #     sock.sendall("CAMERA_MODE_DISABLED1")
      # prev_cam_mode = camera_mode
      sleep(0.01)
except Exception as e:
  print e

\end{lstlisting}

Code block \ref{code:pi_setup} shows the methods to setup Raspberry Pi.

\lstset{language=bash,caption={Raspberry Pi setup},label=code:pi_setup}
\begin{lstlisting}
###############################################################
Enable hostname on windows
###############################################################

$ sudo apt-get update

On the Raspberry Pi you need to install samba and winbind
$ sudo apt-get install samba
$ sudo apt-get install winbind

# to read windows hostname
$ sudo apt-get install libnss-winbind

Edit /etc/nsswitch.conf to enable wins
change 'hosts: files dns' TO 'hosts: files wins dns'

To change the hostname
edit /etc/hostname

FYI 'raspberrypi' works just fine as a host name

$ sudo reboot

###############################################################
Setup arducopter for quad
###############################################################

$ sudo update-alternatives --config arducopter
15

$ sudo nano /etc/default/arducopter 
127.0.0.1 -> 10.0.0.95

$ sudo systemctl daemon-reload

sudo systemctl start arducopter

sudo systemctl enable arducopter

###############################################################
Setup Samba for network drive share
###############################################################

sudo smbpasswd -a pi
sudo smbpasswd -a root
sudo nano /etc/samba/smb.conf

[root_fs]
path = /
valid users = root
read only = no

[pi_fs]
path = /home/pi
valid users = pi
read only = no

sudo service smbd restart

# in windows, win key + R, \\navio


###############################################################
Dronekit setup
###############################################################

sudo pip install dronekit
sudo pip install dronekit-sitl


###############################################################
MAVProxy
###############################################################
sudo apt-get install python-dev python-opencv python-wxgtk3.0 python-pip python-matplotlib python-pygame python-lxml
pip install MAVProxy

# in /etc/rc.local
echo "Starting MAVproxy."
mavproxy.py --out ROD693A:14550 --master localhost:14550 --out localhost:14850 &

###############################################################
PiCamera
###############################################################
sudo apt-get install python-picamera

###############################################################
ssh-key
###############################################################
ssh-keygen -t rsa -C pi@navio
ssh-keygen -t rsa -C pi@infrapi

###############################################################
twisted.internet
###############################################################
pip install twisted


###############################################################
mavlink-router
###############################################################
sudo apt-get install dh-autoreconf
https://github.com/01org/mavlink-router



http://raspberrypi.tomasgreno.cz/ntp-client-and-server.html

sudo apt-get install screen


git remote set-url origin --push --add https://daniel_leonard_robinson@github.com/daniel-leonard-robinson/thesis-image-processing.git
\end{lstlisting}

\end{appendices}

\newpage
\bibliographystyle{plain}
%\bibliographystyle{alpha}
\bibliography{references}

%@misc{gopher,
%	author = {Garden City Collegiate Space Agency},
%    title = {GOPHER SPACE},
%    month = feb,
%    year = {2017},
%    howpublished={\url{http://gopherspace.pbworks.com/w/page/93541286/GOPHER%20SPACE}}
%}
%
%@article{greenwade93,
%    author  = "George D. Greenwade",
%    title   = "The {C}omprehensive {T}ex {A}rchive {N}etwork ({CTAN})",
%    year    = "1993",
%    journal = "TUGBoat",
%    volume  = "14",
%    number  = "3",
%    pages   = "342--351"
%}

\end{document}