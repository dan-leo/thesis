\begin{appendices}

\chapter{Code}

Code block \ref{code:getmatrices} shows the method to get matrices.

\lstset{language=python,caption={Potential packet},label=code:getmatrices}
\begin{lstlisting}
import cv2 #, numpy, matplotlib
import numpy as np
from matplotlib import pyplot as plt
import glob
import dill
from tqdm import tqdm
import os

##img_type = 'ir'
img_type = 'rgb'
##img_type = 'example'

_x = 15
_y = 15

savefile = img_type + '_matrices_15_dual.pkl'
##root_path = 'C:/Users/d7rob/thesis/chess/master_set/'
root_path = 'C:/Users/d7rob/thesis/chess/15_dual_ethernet/rgb/'
##root_path = 'chess/'
##images = glob.glob('chess/*.jpg')
##images = glob.glob('L:/Backups/thesis/chess/rgb/compressed/picked/*.jpg')
##images = glob.glob('C:/Users/d7rob/thesis/chess/master_set/rgb_7x6/*.jpg')
##images = glob.glob(root_path + img_type + '_7x6/*.jpg')
##images = glob.glob(root_path + img_type + '_15x15/*.jpg')
images = glob.glob(root_path + '/*.jpg')

##gray = 0
# termination criteria
criteria = (cv2.TERM_CRITERIA_EPS + cv2.TERM_CRITERIA_MAX_ITER, 30, 0.001)
# prepare object points, like (0,0,0), (1,0,0), (2,0,0) ....,(6,5,0)
objp = np.zeros((_y*_x,3), np.float32)
objp[:,:2] = np.mgrid[0:_x,0:_y].T.reshape(-1,2)
# Arrays to store object points and image points from all the images.
objpoints = [] # 3d point in real world space
imgpoints = [] # 2d points in image plane.

for fname in tqdm(images):
    print fname
    img = cv2.imread(fname)
    gray = cv2.cvtColor(img, cv2.COLOR_BGR2GRAY)
    # Find the chess board corners
    
    flags = cv2.CALIB_CB_ADAPTIVE_THRESH + \
            cv2.CALIB_CB_NORMALIZE_IMAGE + \
            cv2.CALIB_CB_FILTER_QUADS + \
            cv2.CALIB_CB_FAST_CHECK
##    ret, corners = cv2.findChessboardCorners(gray, (_x,_y), flags)

    ret, corners = cv2.findChessboardCorners(gray, (_x,_y), None, flags)
    # If found, add object points, image points (after refining them)
    if ret == True:
        objpoints.append(objp)
        corners2=cv2.cornerSubPix(gray,corners, (11,11), (-1,-1), criteria)
        imgpoints.append(corners2)
        # Draw and display the corners
        cv2.drawChessboardCorners(img, (_x,_y), corners2, ret)
        draw_path = root_path + os.path.basename(fname)[:-4] + '_'+str(_x)+'x'+str(_y)+'_drawn.jpg'
        print draw_path
        cv2.imwrite(draw_path, img)
##        cv2.imshow('img', img)
##        cv2.waitKey(500)
cv2.destroyAllWindows()

ret, mtx, dist, rvecs, tvecs = cv2.calibrateCamera(objpoints, imgpoints, gray.shape[::-1], None, None)
print ret, mtx, dist, rvecs, tvecs
##img = cv2.imread('C:/Users/d7rob/thesis/chess/rgb/img1_2017-09-25_23-18-52.001_1.jpg')
####img = cv2.imread('chess/left12.jpg')
##h,  w = img.shape[:2]
##newcameramtx, roi=cv2.getOptimalNewCameraMatrix(mtx, dist, (w,h), 1, (w,h))

mean_error = 0
for i in xrange(len(objpoints)):
    imgpoints2, _ = cv2.projectPoints(objpoints[i], rvecs[i], tvecs[i], mtx, dist)
    error = cv2.norm(imgpoints[i], imgpoints2, cv2.NORM_L2)/len(imgpoints2)
    mean_error += error
total_error = mean_error/len(objpoints)
print( "total error: {}".format(total_error) )

dill.dump_session(savefile)

\end{lstlisting}

\newpage
Code block \ref{code:sortcorners} shows the method to sort corner pics from unsuccessful ones.

\lstset{language=python,caption={Potential packet},label=code:sortcorners}
\begin{lstlisting}
import glob, cv2
import numpy as np
from matplotlib import pyplot as plt

##images = glob.glob('L:/Backups/thesis/chess/rgb/compressed2/picked/*.jpg')
root_dir = 'C:/Users/d7rob/thesis/chess/15_dual_ethernet/compressed'
images = glob.glob(root_dir + '/*.jpg')
print images

for fname in images:
    img = cv2.imread(fname)

    ##gray = cv2.cvtColor(img, cv2.COLOR_RGB2GRAY)
    gray = cv2.cvtColor(img, cv2.COLOR_BGR2GRAY)
    ##gray = img

    ##plt.imshow(gray)
    ##plt.show()

    ##cv2.imshow('gray', gray)
    ##cv2.waitKey(500)
    ##cv2.destroyAllWindows()

    flags = cv2.CALIB_CB_ADAPTIVE_THRESH + \
            cv2.CALIB_CB_NORMALIZE_IMAGE + \
            cv2.CALIB_CB_FILTER_QUADS + \
            cv2.CALIB_CB_FAST_CHECK
##    flags = cv2.CALIB_CB_FAST_CHECK
    ##flags = None
    # 7, 6
##    ret, corners = cv2.findChessboardCorners(gray, (15,15), flags)
    ret, corners = cv2.findChessboardCorners(gray, (15,15), None, flags)

    print np.shape(corners), ret

    if ret == True:
        pass
##        cv2.imwrite((str(fname[:-4]) + '_has_15x15_corners.jpg'), img)

\end{lstlisting}

\newpage
Code block \ref{code:stereorectification} shows the method to stereorectify.

\lstset{language=python,caption={Potential packet},label=code:stereorectification}
\begin{lstlisting}
import numpy as np
import dill, glob, tqdm

np.set_printoptions(suppress=True)

root_dir = 'C:/Users/d7rob/thesis/distorted'
root_dir = 'L:/Backups/thesis/longridge'

_rgb_name = "/rgb"
_ir_name = "/ir"
rgb_images = glob.glob(root_dir + _rgb_name + '/*.jpg')
ir_images = glob.glob(root_dir + _ir_name + '/*.jpg')
zipped_images = zip(rgb_images, ir_images)

##print ": " + str()
def pp(val):
    print '::\t' + str(val)

class Matrices:
    def __init__(self, total_error, ret, mtx, dist, rvecs, tvecs, imgpoints, objpoints, gray):
        self.total_error = total_error
        self.ret = ret
        self.mtx = mtx
        self.dist = dist
        self.rvecs = rvecs
        self.tvecs = tvecs
        self.imgpoints = imgpoints
        self.objpoints = objpoints
        self.resolution = gray.shape[::-1]

    def print_values(self):
        print 'total_error:\t\t' + str(self.total_error)
        print 'ret:\t\t\t' + str(self.ret)
        print 'mtx.shape:\t\t' + str(self.mtx.shape)
        print 'dist.shape:\t\t' + str(self.dist.shape)
        print 'rvecs.shape:\t\t' + str(np.shape(self.rvecs))
        print 'tvecs.shape:\t\t' + str(np.shape(self.tvecs))
        print 'np.shape(objpoints):\t' + str(np.shape(self.objpoints))
        print 'np.shape(imgpoints):\t' + str(np.shape(self.imgpoints))
        print 'resolution:\t\t' + str(self.resolution)
        print ''
##        print 'imgpoints2.shape:\t' + str(np.shape(imgpoints2))

##savefile = 'rgb_matrices.pkl'
##savefile = 'rgb_matrices_13_print.pkl'
##savefile = 'rgb_matrices_14_dual.pkl'
##savefile = '_matrices_12_ESS.pkl'
savefile = 'rgb_matrices_15_dual.pkl'
dill.load_session(savefile)
rgb = Matrices(total_error, ret, mtx, dist, rvecs, tvecs, imgpoints, objpoints, gray)
rgb.print_values()

##savefile = 'ir_matrices.pkl'
##savefile = 'ir_matrices_13_print.pkl'
##savefile = '_matrices_12_ESS.pkl'
savefile = 'ir_matrices_15_dual.pkl'
dill.load_session(savefile)
ir = Matrices(total_error, ret, mtx, dist, rvecs, tvecs, imgpoints, objpoints, gray)
ir.print_values()

##### find fundamental matrix F
####F, mask = cv2.findFundamentalMat(np.array(rgb.imgpoints[0]), np.array(ir.imgpoints[0]))
####print "F: " + str(F)
####print "mask.shape: " + str(mask.shape)
####print ''
####
##### uncalibrated stereo rectification
####ret, H1, H2 = cv2.stereoRectifyUncalibrated(rgb.imgpoints[0], ir.imgpoints[0], F, rgb.resolution)
####print "stereoRectifyUncalibrated ret: " + str(ret)
####print "H1: " + str(H1)
####print "H2: " + str(H2)
####
####R1 = np.linalg.inv(rgb.mtx)*H1*rgb.mtx
####R2 = np.linalg.inv(ir.mtx)*H2*ir.mtx
####P1 = rgb.mtx
####P2 = ir.mtx
####
####print "R1: " + str(R1)
####print "R2: " + str(R2)
####print "P1: " + str(P1)
####print "P2: " + str(P2)
####print ''

# mono calibration

ret, mtx, dist, rvecs, tvecs = cv2.calibrateCamera(rgb.objpoints, rgb.imgpoints, rgb.resolution, None, None)

# stereo calibration
flags = cv2.CALIB_FIX_ASPECT_RATIO + \
                    cv2.CALIB_ZERO_TANGENT_DIST + \
                    cv2.CALIB_USE_INTRINSIC_GUESS + \
                    cv2.CALIB_SAME_FOCAL_LENGTH + \
                    cv2.CALIB_RATIONAL_MODEL + \
                    cv2.CALIB_FIX_K3 + cv2.CALIB_FIX_K4 + cv2.CALIB_FIX_K5

criteria = (cv2.TERM_CRITERIA_COUNT + cv2.TERM_CRITERIA_EPS, 100, 0.00001)

initCameraMatrix1 = cv2.initCameraMatrix2D(rgb.objpoints, rgb.imgpoints, rgb.resolution, 0);
initCameraMatrix2 = cv2.initCameraMatrix2D(ir.objpoints, ir.imgpoints, ir.resolution, 0);
initDist = np.array([[0]*5])
print "initCameraMatrix1: " + str(initCameraMatrix1)
print "initCameraMatrix2: " + str(initCameraMatrix2)
    
##retval, cameraMatrix1, distCoeffs1, cameraMatrix2, distCoeffs2, R, T, E, F = cv2.stereoCalibrate(rgb.objpoints, rgb.imgpoints, ir.imgpoints, initCameraMatrix1, initDist, initCameraMatrix2, initDist, rgb.resolution, flags, criteria)

retval, cameraMatrix1, distCoeffs1, cameraMatrix2, distCoeffs2, R, T, E, F = cv2.stereoCalibrate(rgb.objpoints, \
                                                                                                 rgb.imgpoints, \
                                                                                                 ir.imgpoints, \
                                                                                                 rgb.resolution,
                                                                                                 initCameraMatrix1, \
                                                                                                 initDist, \
                                                                                                 initCameraMatrix2, \
                                                                                                 initDist, \
                                                                                                 None, None, None, None, \
                                                                                                 criteria, \
                                                                                                 flags)

print "stereoCalibrate retval: " + str(retval)
print "cameraMatrix1: " + str(cameraMatrix1)
print "distCoeffs1: " + str(distCoeffs1)
print "cameraMatrix2: " + str(cameraMatrix2)
print "distCoeffs2: " + str(distCoeffs2)
print "R: " + str(R)
print "T: " + str(T)
print "E: " + str(E)
print "F: " + str(F)
print ''

# stereo rectification
flags = cv2.CALIB_ZERO_DISPARITY
##flags = None

##R1, R2, P1, P2, Q, validPixROI1, validPixROI2 = cv2.stereoRectify(cameraMatrix1, distCoeffs1, cameraMatrix2, distCoeffs2, rgb.resolution, R, T, flags, 1, rgb.resolution)

R1, R2, P1, P2, Q, validPixROI1, validPixROI2 = cv2.stereoRectify(cameraMatrix1, distCoeffs1, cameraMatrix2, distCoeffs2, rgb.resolution, R, T, None, None, None, None, None, flags, 1, rgb.resolution)

print "R1: " + str(R1)
print "R2: " + str(R2)
print "P1: " + str(P1)
print "P2: " + str(P2)
print "Q: " + str(Q)
print "validPixROI1: " + str(validPixROI1)
print "validPixROI2: " + str(validPixROI2)
print ''

# pre-compute undistortion matrices
rgb_mapx, rgb_mapy = cv2.initUndistortRectifyMap(cameraMatrix1, distCoeffs1, R1, P1, rgb.resolution, cv2.CV_32FC1) # 5
ir_mapx, ir_mapy = cv2.initUndistortRectifyMap(cameraMatrix2, distCoeffs2, R2, P2, ir.resolution, cv2.CV_32FC1) # 5

# batch undistortion

for rgb_path, ir_path in tqdm(zipped_images):
    print ''
    rgb_img = cv2.imread(rgb_path)
    dst = cv2.remap(rgb_img, rgb_mapx, rgb_mapy, cv2.INTER_LINEAR)
    rgb_undistort_path = rgb_path[:-4] + '_stereo_undistorted.jpg'
##    print "rgb_undistort_path: " + str(rgb_undistort_path)
##    cv2.rectangle(dst, validPixROI1[:2], validPixROI1[2:],(0, 0, 255),30)
    dst = dst[validPixROI1[1]:validPixROI1[3], validPixROI1[0]:validPixROI1[2]]
    ##$ cv2.imwrite(rgb_undistort_path, dst)

    ir_img = cv2.imread(ir_path)
    dst = cv2.remap(ir_img, ir_mapx, ir_mapy, cv2.INTER_LINEAR)
    ir_undistort_path = ir_path[:-4] + '_stereo_undistorted.jpg'
##    print "ir_undistort_path: " + str(ir_undistort_path)
##    cv2.rectangle(dst, validPixROI2[:2], validPixROI2[2:],(0, 0, 255),30)
    dst = dst[validPixROI2[1]:validPixROI2[3], validPixROI2[0]:validPixROI2[2]]
    ##$ cv2.imwrite(ir_undistort_path, dst)

    # mono output

####    h, w = rgb_img.shape[:2]
####    newcameramtx, roi = cv2.getOptimalNewCameraMatrix(mtx, dist, (w,h), 1, (w,h))
####    print "newcameramtx: " + str(newcameramtx)
####    print "roi: " + str(roi)
####
####    # undistort mono
####    _mapx, _mapy = cv2.initUndistortRectifyMap(mtx, dist, None, newcameramtx, (w,h), cv2.CV_32FC1) # 5
####    _path = rgb_path
####    _img = cv2.imread(_path)
####    _dst = cv2.remap(_img, _mapx, _mapy, cv2.INTER_LINEAR)
####
####    _undistort_path = _path[:-4] + '_mono_undistorted.jpg'
####    cv2.imwrite(_undistort_path, _dst)

\end{lstlisting}

\newpage
Code block \ref{code:duallistmaker} shows the method to create a dual list.

\lstset{language=python,caption={Potential packet},label=code:duallistmaker}
\begin{lstlisting}
import glob

root_dir = 'C:/Users/d7rob/thesis/distorted/undistorted'
root_dir = 'L:/Backups/thesis/longridge'
root_dir = 'C:/Users/d7rob/thesis/home_lab_window_2'

_rgb_name = "/rgb_blue"
_ir_name = "/ir_blue"
rgb_images = glob.glob(root_dir + _rgb_name + '/*.jpg')
ir_images = glob.glob(root_dir + _ir_name + '/*.jpg')

for x, y in zip(rgb_images, ir_images):
    print y + ', ' + x

\end{lstlisting}


Code block \ref{code:tcp_trigger} shows the method to .

\lstset{language=python,caption={Potential packet},label=code:tcp_trigger}
\begin{lstlisting}
import socket, os, threading
from threading import Thread
from time import sleep, time

clients = set()
clients_lock = threading.Lock()
camera_mode = False
cameras_triggered = 0
time_now = 0


def listener(_client, _address):
    global camera_mode, cameras_triggered
    print("Accepted connection from: ", _address)
    cameras_triggered += 1
    with clients_lock:
        clients.add(_client)
    try:
        while True:
            data = _client.recv(32)
            print("received: " + repr(data))
            if not data:
                break
            if "CAMERA_MODE_ENABLED" in data:
                camera_mode = True
                for c in clients:
                    c.sendall("ACK")
            if "CAMERA_MODE_DISABLED" in data:
                camera_mode = False
                for c in clients:
                    c.sendall("ACK")
            if "CAMERA_COMPLETE":
                cameras_triggered += 1
    except Exception as e:
        print(e)
    finally:
        with clients_lock:
            clients.remove(_client)
            _client.close()

count = 0


def camera_process():
    global camera_mode, cameras_triggered, clients, time_now, count
    print("Trigger thread ready.")
    while True:
        if camera_mode and cameras_triggered >= len(clients) and len(clients) > 0:
            cameras_triggered = 0
            with clients_lock:
                print("Sending trigger " + repr(count) + " to " + repr(len(clients)) + " clients; last " + repr(time() - time_now) + ".")
                count += 1
                time_now = time()
                for c in clients:
                    c.sendall("TRIGGER_NOW")
        sleep(0.001)

t = Thread(target=camera_process)
t.daemon = True
t.start()

#host = 'localhost'
host = ''
port = 10017

s = socket.socket()
s.setsockopt(socket.SOL_SOCKET, socket.SO_REUSEADDR, 1)
s.bind((host, port))
s.listen(3)
th = []

try:
    while True:
        print("Server is listening for connections...")
        client, address = s.accept()
        print ("hi")
        t = Thread(target=listener, args=(client, address))
        t.daemon = True
        th.append(t.start())
except KeyboardInterrupt as e:
    print(e)
\end{lstlisting}

Code block \ref{code:init_mav} shows the method to initialize mavlink.

\lstset{language=bash,caption={Potential packet},label=code:init_mav}
\begin{lstlisting}
#!/bin/bash

echo "MAVlink initialisation started."

arg1=
start=true
PREV_IP_GCS=
while true ; do
	sleep 0.5
	IP_GCS="`ping -I wlan0 -c1 rod693a | sed -nE 's/^PING[^(]+\(([^)]+)\).*/\1/p'`"
	if ((${#IP_GCS})); then
		printf "Ground Control Station IP address is %s\n" "$IP_GCS"
		#arg1="--out $IP_GCS:14550"
		arg1="-e $IP_GCS:14550"
		if ! [ "$IP_GCS" = "$PREV_IP_GCS" ] ; then start=true ; fi
		PREV_IP_GCS=$IP_GCS
	else
		unset arg1
	fi
	if [ "$start" = true ] ; then
		#sudo pkill -f "mavproxy.py"
		sudo pkill -f "mavlink-routerd"
		#cmd="mavproxy.py --daemon --master localhost:14550 --out localhost:14850 --out 169.254.169.254:14850 $arg1"
		cmd="/home/pi/mavlink-router/mavlink-routerd 0.0.0.0:14550 -e localhost:14850 -e 169.254.169.254:14850 $arg1"
		echo $cmd
		$cmd &
		start=false
	fi
	sleep 4.5
done

\end{lstlisting}

Code block \ref{code:wifiup} shows the method to reconnect wifi.

\lstset{language=bash,caption={Potential packet},label=code:wifiup}
\begin{lstlisting}
#!/bin/bash

echo "Wifi reconnection monitor started."

while true ; do
  if iwconfig wlan0 | grep -q "ESSID:off" ; then
    echo "Network connection down! Attempting reconnection."
    ifup --force wlan0
  fi
  sleep 5
done

\end{lstlisting}

Code block \ref{code:rc_local} shows the method to boot scripts.

\lstset{language=bash,caption={Potential packet},label=code:rc_local}
\begin{lstlisting}
#!/bin/sh -e
#
# rc.local
#
# This script is executed at the end of each multiuser runlevel.
# Make sure that the script will "exit 0" on success or any other
# value on error.
#
# In order to enable or disable this script just change the execution
# bits.
#
# By default this script does nothing.

# Print the IP address
_IP=$(hostname -I) || true
if [ "$_IP" ]; then
  printf "My IP address is %s\n" "$_IP"
fi

su - pi -c "screen -dmS wifi sudo /home/pi/re_wlan.sh"
su - pi -c "screen -dmS mav ~/init_mav.sh"
su - pi -c "screen -dmS cam python ~/thesis-image-processing/dronekit/capture_mode.py"

exit 0

\end{lstlisting}

Code block \ref{code:interfaces} shows the method interfaces file.

\lstset{language=bash,caption={Potential packet},label=code:interfaces}
\begin{lstlisting}
# interfaces(5) file used by ifup(8) and ifdown(8)

# Please note that this file is written to be used with dhcpcd
# For static IP, consult /etc/dhcpcd.conf and 'man dhcpcd.conf'

# Include files from /etc/network/interfaces.d:
source-directory /etc/network/interfaces.d

auto lo
iface lo inet loopback

allow-hotplug intwifi0
iface intwifi0 inet manual
    wpa-conf /etc/wpa_supplicant/wpa_supplicant.conf

allow-hotplug wlan0
iface wlan0 inet manual
    wpa-conf /etc/wpa_supplicant/wpa_supplicant.conf

allow-hotplug wlan1
iface wlan1 inet manual
    wpa-roam /etc/wpa_supplicant/wpa_supplicant.conf

allow-hotplug eth0
iface eth0 inet manual

iface default inet dhcp

\end{lstlisting}

Code block \ref{code:arducopter} shows the method to connect arducopter.

\lstset{language=bash,caption={Potential packet},label=code:arducopter}
\begin{lstlisting}
# Default settings for ArduPilot for Linux.
# The file is sourced by systemd from arducopter.service

#TELEM1="-A tcp:127.0.0.1:5763:wait"
TELEM1="-A udp:127.0.0.1:14550"
TELEM2="-C /dev/ttyAMA0"
TELEM3="-D udp:10.0.0.95:14550"
#TELEM3="-D udp:0.0.0.0:14250"
TELEM4="-C /dev/ttyUSB0"
TELEM5="-C /dev/ttyUSB1"

# Options to pass to ArduPilot
ARDUPILOT_OPTS="$TELEM1 $TELEM2 $TELEM3" #$TELEM4 $TELEM5"

                          #    # ###### #      #####
                          #    # #      #      #    #
                          ###### #####  #      #    #
                          #    # #      #      #####
                          #    # #      #      #
                          #    # ###### ###### #

# -A is a console switch (usually this is a Wi-Fi link)

# -C is a telemetry switch
# Usually this is either /dev/ttyAMA0 - UART connector on your Navio
# or /dev/ttyUSB0 if you're using a serial to USB convertor

# -B or -E is used to specify non default GPS

# Type "emlidtool ardupilot" for further help
\end{lstlisting}

Code block \ref{code:simultaneous} shows the method to trigger simultaneous shots.

\lstset{language=python,caption={Potential packet},label=code:simultaneous}
\begin{lstlisting}
from dronekit import connect, VehicleMode
from threading import Thread
from time import sleep, time
# import datetime as dt
from datetime import datetime, timedelta
import socket, os, threading, sys
import picamera

UDP_IP = "169.254.169.254"
UDP_PORT = 5005

# Connect to UDP endpoint.
connecting_bool = True
while connecting_bool:
  try:
    print "Connecting to UDP endpoint."
    vehicle = connect('0.0.0.0:14850', wait_ready=True)
    if vehicle: connecting_bool = False
  except KeyboardInterrupt:
    connecting_bool = False
    sys.exit(0)
  except Exception as e:
    print e

camera = picamera.PiCamera()
camera.resolution = (3280, 2464)
camera.start_preview()

print " Mode: %s" % vehicle.mode.name
print vehicle.channels['1']

##print "Autopilot Firmware version: %s" % vehicle.version
##print "Autopilot capabilities (supports ftp): %s" % vehicle.capabilities.ftp
##print "Global Location: %s" % vehicle.location.global_frame
##print "Global Location (relative altitude): %s" % vehicle.location.global_relative_frame
##print "Local Location: %s" % vehicle.location.local_frame    #NED
##print "Attitude: %s" % vehicle.attitude
##print "Velocity: %s" % vehicle.velocity
##print "GPS: %s" % vehicle.gps_0
##print "Groundspeed: %s" % vehicle.groundspeed
##print "Airspeed: %s" % vehicle.airspeed
##print "Gimbal status: %s" % vehicle.gimbal
##print "Battery: %s" % vehicle.battery
##print "EKF OK?: %s" % vehicle.ekf_ok
##print "Last Heartbeat: %s" % vehicle.last_heartbeat
##print "Rangefinder: %s" % vehicle.rangefinder
##print "Rangefinder distance: %s" % vehicle.rangefinder.distance
##print "Rangefinder voltage: %s" % vehicle.rangefinder.voltage
##print "Heading: %s" % vehicle.heading
##print "Is Armable?: %s" % vehicle.is_armable
##print "System status: %s" % vehicle.system_status.state
##print "Mode: %s" % vehicle.mode.name    # settable
##print "Armed: %s" % vehicle.armed    # settable

#@vehicle.on_attribute('location')
#def listener(self, attr_name, value):
    # `self`: :py:class:`Vehicle` object that has been updated.
    # `attr_name`: name of the observed attribute - 'location'
    # `value` is the updated attribute value (a :py:class:`Locations`). This can be queried for the frame information
    #print " Global: %s" % value.global_frame
    #print " GlobalRelative: %s" % value.global_relative_frame
    #print " Local: %s" % value.local_frame

@vehicle.on_message('MAV_CMD_DO_DIGICAM_CONTROL')
def my_method(self, name, msg):
    print " name: %s" % name
    print " msg: %s" % msg

@vehicle.on_message('CAMERA_TRIGGER')
def my_method2(self, name, msg):
    print " name: %s" % name
    print " msg: %s" % msg

last_trigger_time = 0
cam_count = 0
trigger_count = 0
@vehicle.on_message('CAMERA_FEEDBACK')
def process_camera(self, name, msg):
    global last_trigger_time, camera, trigger_count
    trigger_count += 1
    if (trigger_count % 2):
        pass
      #print msg.time_usec
      ##take_photo(msg.time_usec)
    #if time() - last_trigger > 1.0/2:
        #print " name: %s" % name
        #print " msg: %s" % msg
        #print msg.time_usec
    #   last_trigger = time()
    #   missed=0
    #    cam_count += 1
    #else:
    #   if (missed % 2 > 0):
    #       print "missed_" + datetime.datetime.now().strftime('%Y-%m-%d_%H-%M-%S.%f')[:-3]
    #   missed += 1
    
def wait():
    # Calculate the delay to the start of the next hour
    next_sec = (datetime.now() + timedelta(seconds=1)).replace(microsecond=0)
    delay = (next_sec - datetime.now()).microseconds/1000000.0
    # print "delay: " + repr(delay)
    sleep(delay)

def take_photo():
    global cam_count
    cam_count += 1
    dir = '/home/pi/images/'
    os.system("mkdir -p " + dir)
    wait()
    string = dir + 'img_' + datetime.now().strftime('%Y-%m-%d_%H-%M-%S.%f')[:-3] + '_rgb_' + str(cam_count) + '.jpg'
    print string
    #string2 = dir + 'img2_' + time.strftime("%Y-%m-%d_%H-%M-%S_", time.gmtime(t/1000000)) + str(t)[-6:-3] + '_' + str(cam_count) + '.jpg'
    # print string2
    # sleep(0.5)
    camera.capture(string)
    # time.sleep(0.2)

"""
sock = socket.socket()
ack = False
connected = False


def process_data():
    global sock, ack, connected
    while not connected:
        try:
            sock = socket.socket()
            if not sock.connect(('localhost', 10017)):
              connected = True
        except Exception as e:
          print(e)
    try:
        while True:
            data = sock.recv(32)
            # print("data: " + repr(data))
            if "KEEP_ALIVE" in data:
                sock.sendall("KEEP_ALIVE")
            if "ACK" in data:
                ack = True
            if "TRIGGER_NOW" in data and camera_mode:
                take_photo(0)
                sock.sendall("CAMERA_COMPLETE1")
    except Exception as e:
        print(e)
    finally:
        sock.close()
"""

#Create a message listener for all messages.
#@vehicle.on_message('*')
#def listener(self, name, message):
#    print name
#    if 'CAMERA_FEEDBACK' in name:
# print message
#    print 'message: %s' % message

"""
t = Thread(target=process_data)
t.daemon = True
t.start()
"""

camera_mode = False
prev_cam_mode = False
# cam_count = 0

try:
  while True:
      camera_mode = vehicle.channels['8'] > 1100# or True
      if camera_mode:
        take_photo()
      else:
        cam_count = 0
      # if camera_mode != prev_cam_mode:
      #   ack = False
      # if camera_mode:
      #   if not ack and connected:
      #     sock.sendall("CAMERA_MODE_ENABLED1")
      # else:
      #   if not ack and connected:
      #     sock.sendall("CAMERA_MODE_DISABLED1")
      # prev_cam_mode = camera_mode
      sleep(0.01)
except Exception as e:
  print e

\end{lstlisting}

Code block \ref{code:pi_setup} shows the method to setup pi.

\lstset{language=bash,caption={Potential packet},label=code:pi_setup}
\begin{lstlisting}
###############################################################
Enable hostname on windows
###############################################################

$ sudo apt-get update

On the Raspberry Pi you need to install samba and winbind
$ sudo apt-get install samba
$ sudo apt-get install winbind

# to read windows hostname
$ sudo apt-get install libnss-winbind

Edit /etc/nsswitch.conf to enable wins
change 'hosts: files dns' TO 'hosts: files wins dns'

To change the hostname
edit /etc/hostname

FYI 'raspberrypi' works just fine as a host name

$ sudo reboot

###############################################################
Setup arducopter for quad
###############################################################

$ sudo update-alternatives --config arducopter
15

$ sudo nano /etc/default/arducopter 
127.0.0.1 -> 10.0.0.95

$ sudo systemctl daemon-reload

sudo systemctl start arducopter

sudo systemctl enable arducopter

###############################################################
Setup Samba for network drive share
###############################################################

sudo smbpasswd -a pi
sudo smbpasswd -a root
sudo nano /etc/samba/smb.conf

[root_fs]
path = /
valid users = root
read only = no

[pi_fs]
path = /home/pi
valid users = pi
read only = no

sudo service smbd restart

# in windows, win key + R, \\navio


###############################################################
Dronekit setup
###############################################################

sudo pip install dronekit
sudo pip install dronekit-sitl


###############################################################
MAVProxy
###############################################################
sudo apt-get install python-dev python-opencv python-wxgtk3.0 python-pip python-matplotlib python-pygame python-lxml
pip install MAVProxy

# in /etc/rc.local
echo "Starting MAVproxy."
mavproxy.py --out ROD693A:14550 --master localhost:14550 --out localhost:14850 &

###############################################################
PiCamera
###############################################################
sudo apt-get install python-picamera

###############################################################
ssh-key
###############################################################
ssh-keygen -t rsa -C pi@navio
ssh-keygen -t rsa -C pi@infrapi

###############################################################
twisted.internet
###############################################################
pip install twisted


###############################################################
mavlink-router
###############################################################
sudo apt-get install dh-autoreconf
https://github.com/01org/mavlink-router



http://raspberrypi.tomasgreno.cz/ntp-client-and-server.html

sudo apt-get install screen


git remote set-url origin --push --add https://daniel_leonard_robinson@github.com/daniel-leonard-robinson/thesis-image-processing.git
\end{lstlisting}



\chapter{Bill of Materials}

\begin{itemize}
	\item F450 frame
	\item Motors
\end{itemize}

\end{appendices}