% !TEX program = pdflatex
% !BIB program = biber

%% One can optionally have all this inside a separate setup.tex
% !TEX program = pdflatex
% !BIB program = biber

% !TEX root = main.tex
\documentclass[10pt, a4paper]{article}
% \documentclass[12pt, a4paper, oneside]{memoir}
% \chapterstyle{veelo}

%% Sets page size and margins
\usepackage[a4paper,top=3cm,bottom=2cm,left=3cm,right=3cm,marginparwidth=1.75cm]{geometry}
%\usepackage[left=2cm,top=2cm,bottom=2cm,bindingoffset=1cm]{geometry}

\usepackage[utf8x]{inputenc}
\usepackage{graphicx}
\usepackage{float}
\usepackage{imakeidx}
\usepackage{amsmath}
\usepackage[colorlinks=true, allcolors=blue]{hyperref}
\usepackage{graphicx}
\usepackage{float}
\usepackage{imakeidx}
\usepackage{amsmath}
\usepackage{url}
\usepackage[export]{adjustbox}
\usepackage{subcaption}

%% Useful packages
\usepackage[colorinlistoftodos]{todonotes}
\usepackage[T1]{fontenc}

%% Language and font encodings
\usepackage[english]{babel}

%% Define a few colours to be used throughout the document
\usepackage{tikz,xcolor}
\definecolor{TextColor}{HTML}{000000}
\definecolor{SideColorDark}{HTML}{000000}
\definecolor{MainColor}{HTML}{0000FF}
\definecolor{OppositeColor}{HTML}{FF0000}
\definecolor{HighlightColor}{HTML}{FFFF00}


%% Code block style
%  Load the \ttfamily font
\usepackage[T1]{fontenc}
\usepackage[scaled]{beramono}

%  Format code blocks
\usepackage{listings}
%  Change caption name
\renewcommand*{\lstlistingname}{Code block}
\captionsetup[lstlisting]{margin=0cm,format=hang,font=small,format=plain,labelfont={bf,up},textfont={it}}
%  Style
\lstset{
  showstringspaces=false,
  formfeed=\newpage,
  commentstyle=\itshape,
  backgroundcolor=\color{gray!5},
  breakatwhitespace=false,         % sets if automatic breaks should only happen at whitespace
  breaklines=true,                 % sets automatic line breaking
  captionpos=b,                    % sets the caption-position to bottom
  commentstyle=\color{gray},    % comment style
  escapeinside={\%*}{*)},          % if you want to add LaTeX within your code
  keepspaces=true,
  numbersep=2mm,                   % how far the line-numbers are from the code
  showspaces=false,
  showstringspaces=false,
  showtabs=false,
  stepnumber=1, numberfirstline=false,
  basicstyle=\linespread{1}\footnotesize\ttfamily,
  keywordstyle=\bfseries\color{MainColor},
  stringstyle=\itshape\color{OppositeColor},
  numberstyle=\footnotesize\ttfamily\color{gray},
  numbers=left,xleftmargin=4mm,framexleftmargin=0mm,xrightmargin=0mm,
  frame=top,frame=bottom,
}

\title{Insert title here}
\author{Daniel Robinson\\18361137}

\begin{document}
% \maketitle
    \begin{titlepage}
        \begin{center}
            \vspace*{1cm}
            
            \begin{figure}
			\centering
            \includegraphics[scale=2]{images/UScrest-top.jpg}
            \end{figure}
            
            \huge
            \textbf{Normalized Differential Vegetation Index Mapping}

            \large            
            \vspace{2.5cm}

            \textbf{Daniel Robinson\\18361137}

            \vspace{2.5cm}    
            
            \textbf{Report submitted in partial fulfilment of the requirements of the module Project (E) 448
            for the degree Baccalaureus in Engineering in the Department of Electrical and Electronic
            Engineering at the University of Stellenbosch}
            
            \vspace{4cm} 
            
            \textbf{STUDY LEADER: Corné van Daalen\\DATE: November 2017}
            
            
        \end{center}
\end{titlepage}

\newpage
\section{Acknowledgements}
\section{Declaration of own work}

I, the undersigned, hereby declare that the work contained in this report is my own original work
unless indicated otherwise.\\\\

\noindent
Signature: \underline{ }\underline{ } Date: \underline{ }\underline{ }

\section{Summaries}
\subsection{English}
\subsection{Afrikaans}

\tableofcontents
\listoffigures
\listoftables

\newpage

% \begin{abstract}
% Your abstract.
% \end{abstract}

\section{Project Overview}

In this section the project objectives and requirements are specified.

\subsection{Project Objectives}
In order to be deemed successful the project needs to satisfy the following:

\begin{itemize}
    \item Investigate observable parameter(s) beneficial to agriculture within large areas
\end{itemize}

\subsection{Means}

\begin{itemize}
    \item Develop an observation platform to map areas from height (NADIR)
    \item Determine observation equipment
    \item Process sampled data and draw conclusions
    \item Develop control tests to verify / prove conclusions
\end{itemize}

\subsection{Why}

There's currently a water shortage crisis in South Africa, and in other parts of the world, potentially due to global warming. One of the sectors that it affects is agriculture. Farmers may be using less water on their crops, or load sharing. There's also the case of human error, where incorrect amounts of water or fertilizer (for example) is used. Lastly, plant disease is also a problem.\\

\noindent
To mitigate such problems, farmers would physically observe or sample their produce from the ground. It can be a time consuming task, and if sensor data is used, a mass of hand-held readings can prove complex. Perhaps a simpler, large-scale method of observing such changes can be used.\\

\noindent
For some things in the world, to process large amounts of data, it is typically more efficient to use digital equipment than human brain-power. To observe an area in the digital world, one can observe signals from the electromagnetic spectrum.\\

\noindent
Although there exists suitable equipment today, it remains yet inaccessible due to the complexity and relatively high cost of equipment. Perhaps there can be designed a method to bridge this gap.

\subsection{How}

To monitor large areas in a short period of time, one needs height. Height can be achieved by the modern invention of flight. In more practical terms, we can use vehicles such as hot air balloons, airstats, planes, helicopters and the like.\\

\noindent
We can classify these vehicles into manned and unmanned aerial vehicles. Unmanned vehicles have significantly less safety precautions and regulations, cost less, and are less subject to wear and tear, meaning more flights.\\\\

\noindent
We also need to save and process the imagery. Digital equipment utilizing the relevant technologies will suffice. Combining such camera equipment and UAVs, we need a vehicle that will meet the objective.

\subsubsection{Comparison between different types of UAVs}

\begin{itemize}
    \item \textbf{Plane}\\
    Planes are fast, and perhaps too fast. Of course one will be able to optimise the resolution vs speed, but it's not hte best
    \item \textbf{Drone}
    Drones on the other hand are inexpensive, and can take accurate high resolution images, with less risk to the camera equipment than a plan.
\end{itemize}

\section{Setup}

\subsection{Raspberry Pi}

Two Raspberry Pis.

\subsection{Building Drone}

\begin{figure}
\centering
\includegraphics[scale=0.1]{images/drone-build-frame.jpg}
\caption{F450 frame.}
% Reproduced from \cite{gopher}}
\label{fig:frame}
\end{figure}

\begin{figure}
\centering
\includegraphics[scale=0.1]{images/drone-build-motors.jpg}
\caption{Adding the 920kv motors.}
\label{fig:frame}
\end{figure}

\begin{figure}
\centering
\includegraphics[scale=0.1]{images/drone-build-esc-3phaseunconnected.jpg}
\caption{Added ESCs. 3-phase power leads unconnected}
\label{fig:frame}
\end{figure}

\begin{figure}
\centering
\includegraphics[scale=0.1]{images/drone-build-esc-3phaseconnected.jpg}
\caption{The leads are easy to plug/unplug, so that when motor direction is determined, two phase leads can easily be swapped.}
\label{fig:frame}
\end{figure}

\begin{figure}
\centering
\includegraphics[scale=0.1]{images/drone-build-3dcase.jpg}
\caption{3D printed case for Raspberry Pi and Navio.}
\label{fig:frame}
\end{figure}

\begin{figure}
\centering
\includegraphics[scale=0.1]{images/drone-build-3dcase-pi.jpg}
\caption{Putting the Pi in the case.}
\label{fig:frame}
\end{figure}

\begin{figure}
\centering
\includegraphics[scale=0.1]{images/drone-build-3dcase-pi-navio.jpg}
\caption{Fitting the Navio2 flight controller on top.}
\label{fig:frame}
\end{figure}

\begin{figure}
\centering
\includegraphics[scale=0.1]{images/drone-build-3dcase-gps.jpg}
\caption{Connecting the Ublox Neo-7 GPS antenna.}
\label{fig:frame}
\end{figure}

\begin{figure}
\centering
\includegraphics[scale=0.1]{images/drone-build-3dplatform.jpg}
\caption{Stabilising platform. Includes rubber vibration damper balls}
\label{fig:frame}
\end{figure}

\begin{figure}
\centering
\includegraphics[scale=0.1]{images/drone-build-feet.jpg}
\caption{Added feet for platform.}
\label{fig:frame}
\end{figure}

\begin{figure}
\centering
\includegraphics[scale=0.1]{images/drone-build-damper-balls.jpg}
\caption{Damper balls for platform.}
\label{fig:frame}
\end{figure}

\begin{figure}
\centering
\includegraphics[scale=0.1]{images/drone-build-case-ondrone.jpg}
\caption{Attached 3D printed case and platform to drone.}
\label{fig:frame}
\end{figure}

\begin{figure}
\centering
\includegraphics[scale=0.1]{images/drone-build-433.jpg}
\caption{Added 433MHz telemtery to drone.}
\label{fig:frame}
\end{figure}


% \begin{figure}[H]
% \begin{subfigure}{0.5\textwidth}
% \includegraphics[scale=0.3]{3.png}
% \caption{Pushbutton Switch.\\
% Reproduced from \cite{cpi}.}
% \label{fig:p}
% \end{subfigure}
% \begin{subfigure}{0.5\textwidth}
% \centering
% \includegraphics[scale=0.1]{2.png}
% \caption{InvenSense MPU-9250.\\
% Reproduced from\cite{iven}.}
% \label{fig:att}
% \end{subfigure}
% \begin{subfigure}{0.5\textwidth}
% \centering
% \includegraphics[scale=0.3]{1.png}
% \caption{Altimeter Module MS5607.\\ Reproduced from \cite{par}.}
% \label{fig:pres}
% \end{subfigure}
% \caption{Sensors}
% \label{fig:sensors}
% \end{figure}


% In order for the data to be transmitted successfully, it needs to be in a format acceptable by the GCS. The data will be stored in a JSON packet created using a C++ library\cite{ajson}.

% Code block \ref{code:json1} shows a potential data packet.

% \lstset{language=html,caption={Potential packet},label=code:json1}
% \begin{lstlisting}
% {
%   "temperature": "11.3",
%   "pressure": "99.325"
% }
% \end{lstlisting}

% \begin{figure}
% \centering
% \includegraphics[scale=0.35]{flight_path.png}
% \caption{CanSat flight trajectory\\
% Reproduced from \cite{gopher}}
% \label{fig:flight_path}
% \end{figure}




% \subsection{How to add Tables}

% Use the table and tabular commands for basic tables --- see Table~\ref{tab:widgets}, for example. 

% \begin{table}
% \centering
% \begin{tabular}{l|r}
% Item & Quantity \\\hline
% Widgets & 42 \\
% Gadgets & 13
% \end{tabular}
% \caption{\label{tab:widgets}An example table.}
% \end{table}

% \subsection{How to write Mathematics}

% \LaTeX{} is great at typesetting mathematics. Let $X_1, X_2, \ldots, X_n$ be a sequence of independent and identically distributed random variables with $\text{E}[X_i] = \mu$ and $\text{Var}[X_i] = \sigma^2 < \infty$, and let
% \[S_n = \frac{X_1 + X_2 + \cdots + X_n}{n}
%       = \frac{1}{n}\sum_{i}^{n} X_i\]
% denote their mean. Then as $n$ approaches infinity, the random variables $\sqrt{n}(S_n - \mu)$ converge in distribution to a normal $\mathcal{N}(0, \sigma^2)$.


% \subsection{How to create Sections and Subsections}

% Use section and subsections to organize your document. Simply use the section and subsection buttons in the toolbar to create them, and we'll handle all the formatting and numbering automatically.

% \subsection{How to add Lists}

% You can make lists with automatic numbering \dots

% \begin{enumerate}
% \item Like this,
% \item and like this.
% \end{enumerate}
% \dots or bullet points \dots
% \begin{itemize}
% \item Like this,
% \item and like this.
% \end{itemize}

% \subsection{How to add Citations and a References List}

% You can upload a \verb|.bib| file containing your BibTeX entries, created with JabRef; or import your \href{https://www.overleaf.com/blog/184}{Mendeley}, CiteULike or Zotero library as a \verb|.bib| file. You can then cite entries from it, like this: \cite{greenwade93}. Just remember to specify a bibliography style, as well as the filename of the \verb|.bib|.

% You can find a \href{https://www.overleaf.com/help/97-how-to-include-a-bibliography-using-bibtex}{video tutorial here} to learn more about BibTeX.

% We hope you find Overleaf useful, and please let us know if you have any feedback using the help menu above --- or use the contact form at \url{https://www.overleaf.com/contact}!

\newpage
\bibliographystyle{plain}
%\bibliographystyle{alpha}
\bibliography{references}

\end{document}